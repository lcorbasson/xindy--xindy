% $Id$
%----------------------------------------------------------------------

%
% eurotex98.tex
%
% Article to be submitted to EuroTeX'98
%
% [LaTeX]
% (history at end)



\documentclass[10pt,draft]{cah-gut}

\usepackage[isolatin]{inputenc}
\usepackage{rcs}
\usepackage{xspace}
\usepackage{jsdoc}
\usepackage{alltt}
\usepackage{enumerate}
\usepackage{verbatim}
\usepackage{url}
\usepackage{epsfig}


%%%%%%%%%%%%%%%%%%%%%%%%%%%%%%%%%%%%%%%%%%%%%%%%%%%%%%%%%%%%%%%%%%%%%%
%%
%% This has been changed from the cah-gut.cls
%% Comment this section if the style guide shouldn't be changed.
%%

% Use "References":
\renewcommand{\refname}{References}

% Typeset references in \small
\def\thebibliography#1{\section*{\refname}%
\list{[\arabic{enumi}]}{\settowidth\labelwidth{[#1]}\leftmargin\labelwidth
\advance\leftmargin\labelsep\usecounter{enumi}}\small %J.A. 31/07/95
}\let\endthebibliography=\endlist

% Remove the dot after the section numbers:
\makeatletter
\def\@sect#1#2#3#4#5#6[#7]#8{\ifnum #2>\c@secnumdepth
     \def\@svsec{}\else
     \refstepcounter{#1}\edef\@svsec{\csname the#1\endcsname{}\hskip 1em }\fi
     \@tempskipa #5\relax
      \ifdim \@tempskipa>\z@
        \begingroup #6\relax
          \@hangfrom{\hskip #3\relax\@svsec}{\interlinepenalty \@M #8\par}
        \endgroup
       \csname #1mark\endcsname{#7}\addcontentsline
         {toc}{#1}{\ifnum #2>\c@secnumdepth \else
                      \protect\numberline{\csname the#1\endcsname}\fi
                    #7}\else
        \def\@svsechd{#6\hskip #3\@svsec #8\csname #1mark\endcsname
                      {#7}\addcontentsline
                           {toc}{#1}{\ifnum #2>\c@secnumdepth \else
                             \protect\numberline{\csname the#1\endcsname}\fi
                       #7}}\fi
     \@xsect{#5}}
\makeatother

%%
%%
%%%%%%%%%%%%%%%%%%%%%%%%%%%%%%%%%%%%%%%%%%%%%%%%%%%%%%%%%%%%%%%%%%%%%%

\hyphenation{
  strength-en
  Make-In-dex
  }

\def\mkidx{\emph{MakeIndex}\xspace}
\def\Xindy{{\normalfont\textsf{xindy}}\xspace}
\def\imkidx{{\emph{International MakeIndex}}\xspace}
\def\term#1{\emph{#1}}
\def\Lisp{\textsc{Lisp}\xspace}
\def\CL{\textsc{Common Lisp}\xspace}

\def\emdash{--}

\newcommand{\xindy}{%%
   \mbox{\normalfont%%
     \textsf{x\kern-0.6pt%%
       \shortstack{{\scriptsize$\circ$}\\[-2pt]\i}%%
       \kern-1pt%%
       ndy}%%}%%
     }\xspace}

\newcommand{\XINDY}{%
   \mbox{\normalfont\huge%%
         \textsf{x\shortstack{{\large$\circ$}\\\i}ndy}}}


%%%%%%%%%%%%%%%%%%%%%%%%%%%%%%%%%%%%%%%%%%%%%%%%%%%%%%%%%%%%%%%%%%%%%%

\begin{document}

\title{\XINDY{} --- A Flexible Indexing System}
\titlehead{\xindy{} --- A Flexible Indexing System}
\authorhead{Roger Kehr}
\affiliation{Institut f�r Theoretische Informatik\\
  Darmstadt University of Technology\\
  Wilhelminenstra�e 7\\
  D-64283 Darmstadt, Germany\\
  {\normalfont\slshape\texttt{kehr@iti.informatik.tu-darmstadt.de}}}
\author{Roger \textsc{Kehr}}

\RCSdate $Date$

\maketitle

\begin{Abstract}

  Whilst \mkidx is an index processor which is suitable for the
  production of indexes in conjunction with many text formatters, its
  support for non-English languages is weak and a new version called
  \imkidx has been presented, for processing international documents.
  The improvements concentrate on the internationalization of the
  sorting process for keywords in an index. Though it substantially
  improves the availability to sort new languages, there are still
  weaknesses in the processing model largely inherited from \mkidx.
  Through the experiences gained from the \imkidx project we
  implemented a new index processor \xindy that (a) improves the
  sorting of index entries at a finer granularity than \imkidx, (b)
  offers new mechanisms for processing structured location references
  besides page numbers and Roman numerals, and (c) allows for complex
  mark-up schemes.

\end{Abstract}

\begin{Keywords}
  index processor, \mkidx, \imkidx, structured location references,
  context-based mark-up
\end{Keywords}


%%%%%%%%%%%%%%%%%%%%%%%%%%%%%%%%%%%%%%%%%%%%%%%%%%

\section{Existing Systems}

\noindent
The probably most-used index processor in the \TeX{} community is
\mkidx~\cite{Chen:SPE-19-9-897}. It is independent from the document
preparation system, and must be adapted to a particular system using a
configuration file usually referred to as the \term{index style}.
\mkidx merges and sorts \term{index entries}, sorts the \term{location
  references} (such as page numbers), builds ranges (if possible and
desired), and generates the index with mark-up specified in the index
style. Most of its functionality is hard-wired into the application
itself, which is sufficient for most of the typical indexes. It works
well in conjunction with English texts, but if other languages have to
be processed, the sorting model of \mkidx shows its drawbacks. Sorting
index entries is based on the lexicographic order of the keywords. If
this does not match the intended sorting order of the index, \mkidx
cannot be used. The separation of the \term{print key} from the
\term{sort key} offered in \mkidx is only a partial solution that
leads to annoying and error-prone specifications in the document
source. This and other problems are discussed in detail in
\cite{Schrod:CG-10-81}.

The \imkidx~\cite{Schrod:CG-10-81,Schrod:Makeindex30-rep} system is
based on \mkidx and has been enhanced to support user-defined rules
for the specification of language-dependent sorting rules, simplifying
the treatment of non-English languages. Based on the experiences from
the \imkidx project, we have designed and implemented the index
processor \xindy~\cite{xindy:tr,xindy:cbm}. It contributes to the
following aspects of indexing which will be described throughout the
rest of this paper.

\noindent
\textsf{Sorting Model.}\quad \xindy refines the sorting model of
\imkidx to achieve even better support for complex sorting rules. With
this model, the sorting rules for many of the currently spoken
languages can be expressed.

\noindent
\textsf{Structured Location References.}\quad It introduces a clean
model of handling structured location references which were supported
in \mkidx only on an \emph{ad-hoc} basis. This allows for the
specification and correct processing for a large class of existing
enumeration schemes.

\noindent
\textsf{Context-based Mark-Up.}\quad As a result of the new model, the
structure of an index has become more complex and a new mechanism for
specifying mark-up based on context information has been developed.


\begin{comment}
  Our new index processor \xindy~\cite{xindy:tr,xindy:cbm} has been
  designed based on the experiences gained from the \imkidx project,
  demonstrating that further enhancements cannot be easily
  incorporated to the existing \mkidx source. It further refines the
  sorting model of \imkidx to achieve even better support for complex
  sorting rules. It introduces a new model of handling
  \term{structured location references} which were supported in \mkidx
  only on an ad-hoc basis.  As a result of the new model, the
  structure of an index has become more complex and a new mechanism
  for specifying mark-up based on context information has been
  developed. These improvements make \xindy a powerful framework for
  the processing of non-regular indexes.

  In the rest of this paper we introduce each each of the topics
  mentioned above and briefly outline the solutions \xindy offers.
  Since a complete discussion of all features \xindy offers is not
  possible in this article, the reader is encouraged to refer to the
  reports listed in the bibliography and the documentation available
  in the distribution.
\end{comment}

%%%%%%%%%%%%%%%%%%%%%%%%%%%%%%%%%%%%%%%%%%%%%%%%%%

\section{Sorting Index Entries or How to Sort French}

\noindent
The most obvious problems with \mkidx is its lack of support for
sorting index entries of non-English languages. Its sorting mechanism
is essentially based on the ISO-Latin alphabet, which is not adequate
for most languages. The \imkidx system has introduced the concept of
\term{merge} and \term{sort mappings}. These mappings consist of
\term{string rewrite rules} that are applied to a keyword to obtain a
new keyword that is used for merging and sorting purposes.  The
mapping steps are shown in Figure~\ref{fig:mappings}.

\begin{figure}[htbp]
  \hspace*{2.3cm}\input{kmappings.latex}
  \caption{Keyword mappings implemented in \imkidx.}
  \label{fig:mappings}
\end{figure}

The merge mapping is used to \term{normalize} keywords, i.e.\ to
indicate that two different writings for a word should be treated
equally. For example, one can define that the character sequences
`\verb|\"a|' and `\verb|�|' (the former is the \TeX-notation of the
latter) are to be treated equally. After this normalization step,
which merges different index entries from the index into a compound
index entry, the index entries must be sorted. The sort mapping
transfers the merge key into a sort key that reflects the
lexicographic order of the index entry. For example, one possible rule
is to map `\verb|\"a|' onto `\verb|ae|' which is sometimes useful in
German indexes. Hence, the sort rules should be written in such a way
that the resulting sort key reflects the order of the index entry
correctly. This mapping scheme has been implemented in \imkidx.

Though this mechanism is a major improvement over the original \mkidx,
it still does not cover important cases which often appear in
practice. As a running example we sort the French words \emph{cote},
\emph{c�te}, \emph{c�t�}, and \emph{cot�}. The French sorting
rules~\cite{ISO14651}---as well as other language sorting rules---have
the concept of \term{sorting phases} that are applied successively to
obtain a total order on a given set of keywords. The French rules say
that in the first phase the diacritical marks should not be considered
at all, and the non-diacritical counterparts should be used instead.
This means that for the words above there is no distinction in the
first sorting phase at all. In a subsequent phase letters with
diacritical marks have to follow letters without diacritical marks.
Additionally, the lexicographic order is from right to left. This
exactly yields the words in the order shown above. Other languages
such as German also have the concept of sorting phases though they
usually stay with the left-to-right lexicographic order.

From an abstract point of view, the model needs to be enhanced by the
concept of multiple sorting phases and possible variations in the
direction in which the lexicographic order should be processed.
Figure~\ref{fig:xmappings} shows the mapping scheme that is implemented
in \xindy.
\begin{figure}[htbp]
    \hspace*{1.8cm}\input{xmappings.latex}
    \caption{Keyword mappings implemented in \protect\xindy.}
    \label{fig:xmappings}
\end{figure}
It supports the user-defined specification of sort rules for several
independent sort phases. For sets of keywords that are equal in sort
phase $n$, the sort rules of phase $n+1$ are applied to obtain a new
order. This is done successively until a total order on the sort keys
is derived or no more sorting phases remain. To achieve a better
overview on what happens in a sorting phase, \xindy offers means to
debug the keyword mappings in detail.


%%%%%%%%%%%%%%%%%%%%%%%%%%%%%%%%%%%%%%%%%%%%%%%%%%

\section{Location References or How to Sort Bible Verses}

\noindent
One of the initial reasons for the development of \xindy was to study
the inherent structure of location references in an indexing model. We
define a \term{location reference} as the entity that references a
concrete location in a document. Besides page numbers and Roman
numerals one can think of locations such as Bible verses like
\emph{Genesis~1:31}; \emph{Exodus~1:7}; \emph{Leviticus~2:3}. An
index processor must provide solutions to at least two different
aspects of location references: (a) the ability to sort correctly
these location references, and (b) to form \term{ranges} of location
references, if possible and desired. Mathematically speaking, (a)
requests a \term{total order} on the location references, whereas (b)
needs a \term{successor-relationship} for location references. The
total order enables one to sort the location references unambiguously,
and the successor-relationship tests for a potential
\term{joinability} of two location references to form a range.

A closer look reveals that there usually is an inherent structure that
gives information about the two relationships. For example, the
location references \emph{Genesis~1:31}; \emph{Exodus~1:7}; and
\emph{Leviticus~2:3} consist of the name of the book, the number of
the chapter, and the number of the verse. If we need to sort these
location references, we usually sort them in the first phase according
to the book they belong to. The order of the books in the Bible is
fixed and does not follow any lexicographic convention. Inside a book,
sorting is done first according to the chapter number and then
according to the verse number in a chapter. Therefore the structural
entities can be described in form of the three \term{alphabets}:
\begin{center}
  \begin{tabular}{l@{ $\in$ }l}
    \emph{book}    &$\{$\emph{Genesis, Exodus, Leviticus}$\}$\\
    \emph{chapter} & set of Arabic numbers\\
    \emph{verse}   & set of Arabic numbers
  \end{tabular}
\end{center}
\noindent
For each of these alphabets there exists a total order and a
well-known successor-relationship. Hence, sorting index entries is a
problem of lexicographically sorting the components of a location
reference. A structural description in this sense is called a
\term{location class} serving as a template for concrete instances of
this class, in our case concrete Bible verse references.

To enable \xindy to process location references, a definition of the
location classes has to be specified---consisting of a sequence of
alphabets and separators---as encountered in the raw index. As
location references are read, it tries to match the location
references (which are available as a plain string) against the
location classes it knows about, and in case of a match it is able to
decompose the structure into its components. A sample specification of
this location class in \xindy is as follows:\footnote{\xindy uses a \Lisp
  notation for the definition of an index style.}%%
%%
{\small\begin{alltt}
   (define-alphabet "bible-chapters" ("Genesis" "Exodus" "Leviticus"))

   (define-location-class "bible-verses"
       ("bible-chapters" :sep "\verb*| |"
        "arabic-numbers" :sep ":" "arabic-numbers"))
\end{alltt}}

\noindent
The argument \texttt{:sep} declares the following argument to be a
separator. The first description defines the alphabet
\emph{bible-chapters} that consists of the three enumerated
letters\footnote{A letter is basically a sequence of characters of the
  underlying document alphabet.} \emph{Genesis}, \emph{Exodus}, and
\emph{Leviticus}. The second definition composes a location class
named \emph{bible-verses} based on the new alphabet, the separator
characters (solely used for matching the location references), and the
built-in alphabet of Arabic numbers. This description essentially
defines the \term{grammar} of the verses occurring in the input. This
description enables \xindy to correctly sort all instances of the
class \emph{bible-verses} and additionally enables it to join location
references into ranges, if desired. \xindy basically allows for the
definition of new alphabets and location classes, which may be of
variable length (such as 1, 1.1, 1.1.1, \ldots) as well. It offers a
wide range of specifications how to join location references to form
ranges.

A new concept called \term{location reference attributes} can be used
to tag location references with additional information that declares a
location reference to be of a particular type, such as a reference to
a definition of an item, another for its occurrence, and so on. \mkidx
introduced the concept of \term{encapsulators} for mark-up purposes.
We have generalized this concept to offer more flexibility in the
sorting and merging phases, for example to indicate that a definition
of an item should subsume its occurrence on the same page, to save
space on the resulting output. We give an example of the possibilities
to output a sequence of page numbers tagged as definitions and
occurrences based on different policies, such as separation of
attributes, building ranges where possible, and subsuming occurrences
of location references with a definition on the same page.
Table~\ref{tab:policies} illustrates some of the possibilities that
can be specified with \xindy (definitions of items are shown in
boldface).
%%
\def\tem#1{\emph{#1}}%%
\begin{table}[htbp]
  \begin{small}
    \begin{tabular}{rl}
      \tem{separate attributes, no ranges} &
      11 13 14 17\ \textbf{12 15 25}\\%%
      \tem{mixed attributes, no ranges} &
      11 \textbf{12} 13 14 \textbf{15} 17 \textbf{25}\\%%
      \tem{mixed attributes, ranges, not subsumed} &
      11 \textbf{12} 13--15 \textbf{15} 17 \textbf{25}\\%%
      \tem{separate attributes, ranges, subsumed} &
      11--15 17\ \textbf{12 15 25} \\%%
      \tem{separate, ranges, subsumed and ommitted} &
      11--15 17\ \textbf{25}\\%%
    \end{tabular}
 \end{small}
  \caption{Output of location references using different policies.}
  \label{tab:policies}
\end{table}

We hope that one might get an impression of what kind of processing
location references with \xindy is possible in general and what
different levels of compression in the resulting output can be
achieved.


%%%%%%%%%%%%%%%%%%%%%%%%%%%%%%%%%%%%%%%%%%%%%%%%%%

\section{Context-Based Mark-Up}

\noindent
In addition to the indexing features introduced, there is a need for a
general model to specify mark-up easily. The set of available mark-up
tags is relatively limited and fixed in \mkidx. In \xindy, the final
index, after all processing steps have been performed, is internally
represented as a tree. Mark-up is implemented with a tree-traversal
algorithm that starts at the root node and visits each node of the
tree in a depth-first manner. Every time a node is entered or left, an
\term{event} is generated. The user is now able to bind mark-up tags
to each of these events. For example, the binding

\begin{small}
  \verb|   (markup-locref :class "bible-verses" :open "\\textit{" :close "}")|
\end{small}

\noindent
denotes that the location references of class \emph{bible-verses}
should be surrounded by the mark-up tags `\verb|\textit{|' and
  `\verb|}|' defining an italicized \TeX{}-mark-up. If the parameter
\verb|:class| would have been omitted, this specification would match
all location classes, therefore acting as a default setting.  If a
mark-up event is raised, the \term{event dispatcher} is responsible
for finding the most specific binding that matches this event. Events
are parameterized by information from the context in which they were
raised.  For example, the events for location references contain
information about the current location class, the current attribute,
and the depth it is placed in. Bindings can be defined to any subset
of the set of parameters. The tag \verb|:open| is the string to be
emitted if a node is entered, whereas \verb|:close| defines the
corresponding binding if a node is left.

At a first glance this scheme sounds more complicated than it is in
practice. Debugging facilities exist to help users specifying mark-up
bindings. The whole mark-up phase can be traced, events are shown, and
the bindings can be seen when they are activated. Usually only a small
portion of all possible events actually need bindings.  Just to give
an impression, Table~\ref{tab:markup} illustrates what results can be
specified with \xindy by mark-up bindings only.

\begin{table}[htbp]
  {\small
    \begin{tabular}{rl}
      \tem{standard tagging} &
      \emph{A.1, A.3, A.7, B.5, B.12}\\%%
      \tem{emphasizing the chapters} &
      \emph{\textbf{A}.1, \textbf{A}.3, \textbf{A}.7, \textbf{B}.5,
        \textbf{B}.12} \\%%
      \tem{additional compression of sections} &
      \emph{\textbf{A} 1,3,7; \textbf{B} 5,12} \\%%
      \tem{standard tagging} &
      \emph{Genesis~1:31}; \emph{Exodus~1:7} \\%%
      \tem{different mark-up for verses} &
      \emph{Genesis~\textbf{1}(31)}; \emph{Exodus~\textbf{1}(7)} \\%%
      \tem{verbose mark-up} &
      \emph{Genesis~chap.\,1,~31}; \emph{Exodus~chap.\,1,~7} \\%%
    \end{tabular}
  }
  \caption{Output of location references with different mark-up.}
  \label{tab:markup}
\end{table}

More examples and detailed descriptions, illustrating how these
results can be obtained, are described in the documentation that comes
with \xindy.


%%%%%%%%%%%%%%%%%%%%%%%%%%%%%%%%%%%%%%%%%%%%%%%%%%

\section{ Implementation, Availability and Distribution}

\noindent
\xindy is largely implemented in \CL. We have chosen the freely
available CLISP-implementation\footnote{Available at
  \texttt{ftp://ftp2.cons.org/pub/lisp/clisp/source/}.},
and have extended it with the GNU \texttt{rx} regular expression
library\footnote{Available at
  \texttt{ftp://prep.ai.mit.edu/pub/gnu/}.} for the keyword mappings.
It consists of about 4\,500 lines of \Lisp code and 600 lines of
\textsf{C} code. A parser for the transformation from the
\TeX{}-specific raw index in the format used by \xindy has been
implemented using 150 lines of \texttt{lex} code. As a comparison,
\mkidx is written in 4\,300 lines of \textsf{C}.

The full implementation is available under the conditions of the GNU
General Public License. Its home-page with further links is accessible
at our web site
\url{http://www.iti.informatik.tu-darmstadt.de/xindy/}. Source and
binary distributions are available at CTAN in directory
\url{pub/indexing/xindy/}. It is currently available in source and
binary distributions for several Unix platforms and OS/2. Efforts to
port \xindy to Windows\,95/NT platforms are underway and are likely to
be finished soon. There is lots of documentation available in various
formats and more detailed examples describe how to use \xindy,
especially its treatment of sorting index entries and managing
location references.


\section{Conclusion}

\noindent
\xindy is a new index processor that improves three major aspects of
indexing. It offers new means for the specification of sorting rules
for the index entries covering languages from English (with its simple
sorting model) to French (with rather complex rules). Users will find
it a valuable tool that allows them to process indexes in their native
language.

Another improvement is achieved by the concept of location references
and attributes that allows structured references (such as Bible verses
or chapter/page enumeration schemes) to be processed, sorted, and
joined in various ways. This is accompanied by a powerful
context-based mark-up scheme that offers very fine control over the
process of tagging the final index, suitable for different mark-up
languages such as \TeX, and instances of SGML or XML documents.

From the user interface perspective, the new module scheme allows for
the development of index styles by authors that can be reused by
end-users with minimal effort. Authors are welcome to contribute to
this project by writing modules for different languages and mark-up
for different back-ends.


%%%%%%%%%%%%%%%%%%%%%%%%%%%%%%%%%%%%%%%%%%%%%%%%%%

\section*{Acknowledgments}

\noindent
I would like to thank Joachim Schrod and Klaus Guntermann for their
inspiring discussions essentially in early phases of this project.
Further, I would like to thank Gabor Herr, who was as excellent
adviser in many implementation questions. Additionally, I would like
to thank the participants in our discussions on the \xindy mailing
list, most notably Chris Rowley, and finally Ulrich Gr�f and
Prof.~Waldschmidt for giving valuable hints for the improvement of
this paper.


\bibliography{bibliographie}
\bibliographystyle{abbrv}

\end{document}


%%%%%%%%%%%%%%%%%%%%%%%%%%%%%%%%%%%%%%%%%%%%%%%%%%%%%%%%%%%%%%%%%%%%%%

$Log$
Revision 1.4  1998/01/28 08:37:36  kehr
Final version submitted to EuroTeX'98.

Revision 1.3  1998/01/28 08:15:09  kehr
Further modifications.

Revision 1.2  1998/01/21 10:20:40  kehr
First submitted version.

Revision 1.1  1997/11/25 19:06:08  kehr
Initial checkin of both texts.

