% $Id$
%----------------------------------------------------------------------

%
% eurotex98.tex
%
% Article to be submitted to EuroTeX'98
%
% [LaTeX]
% (history at end)



\documentclass[10pt,draft]{cah-gut}

\usepackage[isolatin]{inputenc}
\usepackage{rcs}
\usepackage{xspace}
\usepackage{jsdoc}
%\usepackage{ssqquote}
%\usepackage{ifthen}
%\usepackage{theorem}
%\usepackage{bslash}
\usepackage{alltt}
\usepackage{enumerate}
\usepackage{verbatim}
\usepackage{url}
%\usepackage{afterpage}
%\usepackage{rotating}
\usepackage{epsfig}
%\usepackage[hang,sf]{caption}


%%%%%%%%%%%%%%%%%%%%%%%%%%%%%%%%%%%%%%%%%%%%%%%%%%%%%%%%%%%%%%%%%%%%%%
%%
%% This has been changed from the cah-gut.cls
%% Uncomment this section if the style guide shouldn't be changed.
%%

% Use "Keywords":
\renewcommand{\keywords}{\quotation\noindent\small\textbf{Keywords}:
\textrm\bgroup}
\def\endkeywords{\egroup\endquotation \normalsize}

% Use "References":
\renewcommand{\refname}{References}

% Typeset references in \small
\def\thebibliography#1{\section*{\refname}%
\list{[\arabic{enumi}]}{\settowidth\labelwidth{[#1]}\leftmargin\labelwidth
\advance\leftmargin\labelsep\usecounter{enumi}}\small %J.A. 31/07/95
}\let\endthebibliography=\endlist

% Remove the dot after the section numbers:
\makeatletter
\def\@sect#1#2#3#4#5#6[#7]#8{\ifnum #2>\c@secnumdepth
     \def\@svsec{}\else
     \refstepcounter{#1}\edef\@svsec{\csname the#1\endcsname{}\hskip 1em }\fi
     \@tempskipa #5\relax
      \ifdim \@tempskipa>\z@
        \begingroup #6\relax
          \@hangfrom{\hskip #3\relax\@svsec}{\interlinepenalty \@M #8\par}
        \endgroup
       \csname #1mark\endcsname{#7}\addcontentsline
         {toc}{#1}{\ifnum #2>\c@secnumdepth \else
                      \protect\numberline{\csname the#1\endcsname}\fi
                    #7}\else
        \def\@svsechd{#6\hskip #3\@svsec #8\csname #1mark\endcsname
                      {#7}\addcontentsline
                           {toc}{#1}{\ifnum #2>\c@secnumdepth \else
                             \protect\numberline{\csname the#1\endcsname}\fi
                       #7}}\fi
     \@xsect{#5}}
\makeatother

%%
%%
%%%%%%%%%%%%%%%%%%%%%%%%%%%%%%%%%%%%%%%%%%%%%%%%%%%%%%%%%%%%%%%%%%%%%%

\hyphenation{
  strength-en
  Make-In-dex
  }

\def\mkidx{\emph{MakeIndex}\xspace}
\def\Xindy{{\normalfont\textsf{xindy}}\xspace}
\def\imkidx{{\emph{International MakeIndex}}\xspace}
\def\term#1{\emph{#1}}
\def\Lisp{\textsc{Lisp}\xspace}
\def\CL{\textsc{Common Lisp}\xspace}

\def\emdash{--}

\newcommand{\xindy}{%%
   \mbox{\normalfont%%
     \textsf{x\kern-0.6pt%%
       \shortstack{{\scriptsize$\circ$}\\[-2pt]\i}%%
       \kern-1pt%%
       ndy}%%}%%
     }\xspace}

\newcommand{\XINDY}{%
   \mbox{\normalfont\huge%%
         \textsf{x\shortstack{{\large$\circ$}\\\i}ndy}}}


%%%%%%%%%%%%%%%%%%%%%%%%%%%%%%%%%%%%%%%%%%%%%%%%%%%%%%%%%%%%%%%%%%%%%%

\begin{document}

\title{\XINDY{} --- A Flexible Indexing System}
\titlehead{\xindy{} --- A Flexible Indexing System}
\authorhead{Roger Kehr}
\affiliation{Institut f�r Theoretische Informatik\\
  Darmstadt University of Technology\\
  Wilhelminenstra�e 7\\
  D-64283 Darmstadt, Germany\\
  {\normalfont\slshape\texttt{kehr@iti.informatik.tu-darmstadt.de}}}
\author{Roger \textsc{Kehr}}

\RCSdate $Date$

\maketitle

\begin{abstract}
  \mkidx is an index processor which is suitable for the production of
  indexes with many text formatters.  Its support for non-English
  languages is weak and a new version called \imkidx has been
  presented, that offers support for processing international
  documents.  The improvements in \imkidx concentrate on the
  internationalization of the sorting process for keywords in an
  index. Though it does greatly improve the availability to sort new
  languages, there are still weaknesses in the processing model
  largely inherited from \mkidx. Through the experiences gained from
  the \imkidx project we decided to implement a new index processor
  from scratch that (a) improves the sorting of index entries at a
  finer granularity than \imkidx, (b) offers new mechanisms for
  processing structured location references besides page numbers and
  Roman numerals, and (c) allows for complex markup schemes that were
  not possible with the other processors.

\end{abstract}

\begin{keywords}
  index processor, \mkidx, \imkidx, structured location references,
  context-based markup
\end{keywords}


%%%%%%%%%%%%%%%%%%%%%%%%%%%%%%%%%%%%%%%%%%%%%%%%%%

\section{Existing Systems}

\noindent
The probably most-used index processor in the \TeX{} community is
\mkidx~\cite{Chen:SPE-19-9-897}. It is independent from the document
preparation system, and must be adopted to a particular system using a
configuration file called \term{index style}. \mkidx merges and sorts
\term{index entries}, sorts the \term{location references} (such as
page numbers), builds ranges if possible, and outputs the index with
markup specified in the index style. Most of its functionality is
hard-wired into the application itself, which is usually sufficient
for most of the typical indexes. It works well in conjunction with
English texts but if indexes in other languages have to be processed,
the sorting model of \mkidx shows its drawbacks. Sorting index entries
is based on the lexicographic order of the keywords which is a
problem, if this representation does not fit the intended sorting
order. The separation of the \term{print key} from the \term{sort key}
offered in \mkidx is only a partial solution that leads to annoying
and error-prone specifications in the document source. This and other
problems are discussed in detail in \cite{Schrod:CG-10-81}.

The \imkidx~\cite{Schrod:CG-10-81,Schrod:Makeindex30-rep} system is
based on \mkidx and has been enhanced to support user defined rules
for the specification of language-dependent sorting rules, simplifying
the treatment of non-English languages.

Our new index processor \xindy~\cite{xindy:tr,xindy:cbm} has been
designed based on the experiences gained from the \imkidx project,
demonstrating that further enhancements cannot be easily incorporated
to the existing \mkidx source. It further refines the sorting model of
\imkidx to achieve even better support for complex sorting rules. It
introduces a new model of handling \term{structured location
  references} which were supported in \mkidx only on an ad-hoc basis.
As a result of the new model, the structure of an index has become
more complex and a new mechanism for specifying markup based on
context information has been developed. These improvements make \xindy
a powerful framework for the processing of non-regular indexes.

In the rest of this paper we introduce each each of the topics
mentioned above and briefly outline the solutions \xindy offers. Since
a complete discussion of all features \xindy offers is not possible in
this article, the reader is encouraged to refer to the reports listed
in the bibliography and the documentation available in the
distribution.


%%%%%%%%%%%%%%%%%%%%%%%%%%%%%%%%%%%%%%%%%%%%%%%%%%

\section{Sorting Index Entries}

\noindent
The most obvious problems with \mkidx is its lack of support for
sorting of index entries for non-English languages. Its sorting model
is essentially based on the ISO-Latin alphabet, which is not an
adequate sorting order for most languages. The \imkidx system has
introduced the concept of \term{merge} and \term{sort mappings}. These
mappings consist of \term{string rewrite rules} that are applied to a
keyword to obtain a new keyword that is used for merging and sorting
purposes. The mapping steps are shown in Figure~\ref{fig:mappings}.

\begin{figure}[htbp]
  \hspace*{2.3cm}\input{kmappings.latex}
  \caption{Keyword mappings implemented in \imkidx.}
  \label{fig:mappings}
\end{figure}

The merge mapping is used to \term{normalize} keywords, i.e.\ indicate
that two different writings of a word should be treated equally. For
example, one can define that the character sequences `\verb|\"a|' and
`\verb|�|' are to be treated equally. After this normalization step,
that merges different index entries from the index into a compound
index entry, the index entries must be sorted. The sort mappings map
the merge key onto a sort key that determines the lexicographic order
of the index entry. For example, one possible rule is to map
`\verb|\"a|' onto `\verb|ae|' which is sometimes useful in German
indexes. Hence, the sort rules should be written in such a way that
the resulting sort key reflects the order of the index entry
correctly. This mapping scheme has been implemented in \imkidx.

Though this mechanism is a major improvement over the original \mkidx,
it still does not cover all cases that appear in practice. As a
running example we sort the French words \emph{cote}, \emph{c�te},
\emph{c�t�}, and \emph{cot�}. The French sorting
rules~\cite{ISO14651}---as well as other language sorting rules---have
the concept of \term{sorting phases} that are applied successively to
obtain a total order on a given set of keywords. The rules say, that
in a first phase the diacritical marks should not be considered at
all, and the non-diacritical counterparts should be used instead. This
means for the words above that there is no distinction in a first
sorting phase at all. They are treated equally but they are placed
lexicographically somewhere between \emph{cotd} and \emph{cotf}. In
another run there is an order on the diacritical marks specifying that
letters with diacritical marks follow letters without diacritical
marks. Additionally, the lexicographic order is from right to left.
This exactly yields the words in the order shown above. Other
languages such as German as well have the concept of sorting phases
though they usually stay with the left-to-right lexicographic order.

From an abstract point of view the model needs to be enhanced by the
concept of multiple sorting phases and possible variations in the
direction in which the lexicographic order should be processed.
Figure~\ref{fig:xmappings} shows the mapping scheme that is implemented
in \xindy.
\begin{figure}[htbp]
    \hspace*{1.8cm}\input{xmappings.latex}
    \caption{Keyword mappings implemented in \protect\xindy.}
    \label{fig:xmappings}
\end{figure}
It supports the user-defined specification of sort rules for several
independent sort phases. For any set of keywords that are equal in a
certain sort phase $n$, the sort rules of phase $n+1$ are applied to
obtain a new order. This is done successively until a total order on
the sort keys is derived or no sorting phases remain open. To achieve
a better understanding on what happens in a sorting phase, \xindy
offers means to debug the keyword mappings in detail.

For a detailed discussion the reader may refer the documentation that
comes with \xindy.


%%%%%%%%%%%%%%%%%%%%%%%%%%%%%%%%%%%%%%%%%%%%%%%%%%

\section{Location References or How to Sort Bible Verses}

\noindent
One of the initial reasons for the development of \xindy was to study
the inherent structure of location references in an indexing model. A
\term{location reference} is the entity that references a particular
location in a document. Besides page numbers and Roman numerals one
can think of locations such as Bible verses like \emph{Genesis~1,~31};
\emph{Exodus~1,~7}; \emph{Leviticus~2,~3}. An index processor must
provide solutions to at least two different aspects of location
references: (a) the ability to correctly sort these location
references, and (b) to form \term{ranges} of location references if
possible and desired from the user. Mathematically speaking, (a) needs
for a \term{total order} on the location references, whereas (b) needs
for a well-defined \term{successor-relationship} among location
references. The total order enables to sort the location references
unambiguously, and the successor-relationship tests for a potential
\term{joinability} of two location references to form a range.

A closer look at location references shows that they usually have an
inherent structure that gives information about the two relationships
mentioned previously. For example, the location references
\emph{Genesis~1,~31}; \emph{Exodus~1,~7}; and \emph{Leviticus~2,~3}
consist of the name of the book, the number of the chapter, and the
number of the verse. If we need to sort these location references we
usually sort them in a first phase according to the book they belong
to. The order of the books in the bible is fixed and does not follow
any lexicographic convention. Inside a book, sorting is done first
according to the chapter number and then according to the verse number
inside a chapter. The structural entities these locations consist of
can be described in the form of the following three \term{alphabets}:
\begin{center}
  \begin{tabular}{l@{ $\in$ }l}
    \emph{book}    &$\{$\emph{Genesis, Exodus, Leviticus}$\}$\\
    \emph{chapter} & set of Arabic numbers\\
    \emph{verse}   & set of Arabic numbers
  \end{tabular}
\end{center}
\noindent
For each of these alphabets there exists a well-defined total
order---for the set of Arabic numbers there exists a well-defined
successor-relationship as well. Hence, sorting index entries is a
problem of lexicographically sorting the components a location
reference consists of. A structural description in this sense is
called a \term{location class} since it serves as a template for
concrete instances of this class, in our case concrete Bible verse
references.

To enable \xindy to process location references it must be given a
definition of the location classes---consisting of a sequence of
alphabets and separators---it may encounter in the raw index. As
location references are read, it tries to match the location
references (which are available as a plain string) against the
location classes it knows about, and in case of a match it is able to
decompose the structure into its components. A sample specification of
this
location class in \xindy is as follows:%%
%%
{\small\begin{alltt}
   (define-alphabet "bible-chapters" ("Genesis" "Exodus" "Leviticus"))

   (define-location-class "bible-verses"
       ("bible-chapters" :sep\footnote{The argument \texttt{:sep} declares the following argument to be a separator.} "\verb*| |"
        "arabic-numbers" :sep "," "arabic-numbers"))
\end{alltt}}

\noindent
The first description defines the new alphabet \emph{bible-verses}
that consists of the three enumerated \term{letters}\footnote{A letter
  is actually any sequence of characters of the underlying document
  alphabet.} \emph{Genesis}, \emph{Exodus}, and \emph{Leviticus}. The
second definition composes a location class based on the new alphabet,
the separator characters, solely used for matching the location
references, and the built-in alphabet of Arabic numbers. This
description now enables \xindy to correctly sort all instances of the
class \emph{bible-verses} and additionally enables it to join location
references into ranges, if desired. \xindy basically allows for the
definition of new alphabets and location classes, which may be of
variable length (such as 1, 1.1, 1.1.2, \ldots) as well. It offers a
wide range of specifications how to join location references to form
ranges.

A new concept called \term{location reference attributes} can be used
to tag location references with additional information that declares a
location reference to be of a particular type, such as a reference to
a definition of a term, another for its occurrence, and so on. This is
especially important in indexes for manuals. \mkidx had a concept
called \term{encapsulators} which were used for markup purposes only.
We have generalized this concept to offer more flexibility in the
sorting and merging phases, for example to indicate that a definition
of a term should subsume its occurrence on the same page, to save
space on the resulting output. To give an example of the possibilities
to output a sequence of page numbers tagged as definitions (boldface)
and occurrences based on policies such as separation of attributes,
building ranges where possible, and subsumption of occurrences of
location references with a definition on the same page the following
sequences can be produced with \xindy:
\begin{center}
  \begin{tabular}{cc}
    11 13 14 17\ \textbf{12 15 25} & 11 \textbf{12} 13 14 \textbf{15} 17 \textbf{25}\\
    11 13--15 17\ \textbf{12 15 25} & 11 \textbf{12} 13--15 \textbf{15} 17 \textbf{25}\\
    11--15 17\ \textbf{12 15 25} & 11--15 \textbf{12 15} 17 \textbf{25}\\
    11--15 17\ \textbf{25} &  11 13 14 17 12 15 25
  \end{tabular}
\end{center}
We hope that one might get an impression what kind of processing
location references with \xindy is possible in general.


%%%%%%%%%%%%%%%%%%%%%%%%%%%%%%%%%%%%%%%%%%%%%%%%%%

\section{Context-Based Markup}

\noindent
Hand in hand with the new indexing features \xindy offers, there is
also a need to allow for a general model to specify markup easily. The
set of available markup tags is relatively limited and fixed in
\mkidx. In \xindy the final index, after all processing steps have
been done, is internally represented as a tree. Markup is implemented
with a tree-walking algorithm that starts at the root node and visits
each node of the tree in a depth-first manner. Every time a node is
entered or left an \term{event} is generated. The user is now able to
bind markup tags to each of these events. For example, the binding
\begin{verbatim}
      (markup-locref :class "bible-verses"
                     :open "\\textit{" :close "}")
\end{verbatim}
denotes that the location references of class \emph{bible-verses}
should be surrounded by the markup tags `\verb|\textit{|' and
  `\verb|}|' defining an italicized \TeX{}-markup. If the parameter
\verb|:class| would have been omitted, this specification would match
all location classes, therefore acting as a default setting. If a
markup event is raised, the \term{event dispatcher} is responsible to
find the most specific binding matching this event. Events are
parameterized by information from the context in which it was raised.
For example, the events for location references contain information
about the current location class, the current attribute, and the depth
it is placed in. Bindings can be defined to any subset of the set of
parameters. The tag \verb|:open| is the string to be emitted if a node
is entered, whereas \verb|:close| defines the corresponding binding if
a node is left.

At a first glance this scheme sounds more complicated than it is in
practice. Debugging facilities exist to aid the user in the
specification of markup bindings. The whole markup phase can be
traced, events are shown, and the bindings can be seen as they are
activated. Usually only a small portion of all possible events
actually need bindings. Just to give an impression the following
results can be specified with \xindy by markup bindings only:
\begin{center}
  \begin{tabular}{cc}
    \emph{A.1, A.3, A.7, B.5, B.12} & \emph{A 1,3,7; B 5,12}\\%%
    \emph{\textbf{A}.1, \textbf{A}.3, \textbf{A}.7, \textbf{B}.5, \textbf{B}.12} &
    \emph{\textbf{A} 1,3,7; \textbf{B} 5,12}\\
    \emph{Genesis~1,~31}; \emph{Exodus~1,~7} &
    \emph{Genesis~1(31)}; \emph{Exodus~1(7)} \\
    \emph{Genesis~chap.\,1,~31}; \emph{Exodus~chap.\,1,~7} &
    \emph{Genesis~\textbf{1}(31)}; \emph{Exodus~\textbf{1}(7)} \\
  \end{tabular}
\end{center}
\noindent
Again we hope that the reader might get an impression what flexibility
\xindy offers in the processing and tagging of indexes. More examples
and detailed descriptions, how these results can be obtained, are
described in the documentation that comes with \xindy.


%%%%%%%%%%%%%%%%%%%%%%%%%%%%%%%%%%%%%%%%%%%%%%%%%%

\section{ Implementation, Availability and Distribution}

\noindent
\xindy is largely implemented in \CL. We have chosen the freely
available CLISP-implementation\footnote{Available at
  \texttt{ftp://ma2s2.mathematik.uni-karlsruhe.de/pub/lisp/clisp/}.},
and have extended it with the GNU \texttt{rx} regular expression
library\footnote{Available at
  \texttt{ftp://prep.ai.mit.edu/pub/gnu/}.} for the keyword mappings.

It consists of about 4\,500 lines of \Lisp code and 600 lines of
\textsf{C} code. A parser for the transformation from the
\TeX{}-specific raw index in the format used by \xindy has been
implemented using 150 lines of \texttt{lex} code. As a comparison,
\mkidx is written in 4\,300 lines of \textsf{C}.

The full implementation is available under the conditions of the GNU
General Public License. Its home-page with further links is accessible
at our web site
\url{http://www.iti.informatik.tu-darmstadt.de/xindy/}. Source and
binary distributions are available at CTAN in directory
\url{pub/indexing/xindy/}. It is currently available in source and
binary distributions for several Unix platforms and OS/2. Other
platforms are to be supported soon. There is lots of documentation
available in various formats and more detailed examples describe how
to use \xindy, especially its treatment of sorting index entries and
managing location references.


\section{Conclusion}

\noindent
\xindy is an index processor that can be seen as a framework for the
production of high-quality indexes, which first must be tailored for a
certain task, such as sorting a French dictionary or a German Bible.

It offers potential users many ways to sort indexes correctly in
unambiguous ways, allows for the processing of structured location
references, and additionally offers a very powerful markup back-end
tailorable for many document preparation systems.

Since a detailed specification involves many aspects and can be of
considerable length, \xindy offers a module mechanism that allows for
the inclusion of specific rule sets or location class descriptions.
This mechanism makes it easy to master this system even for
unexperienced users.

%%%%%%%%%%%%%%%%%%%%%%%%%%%%%%%%%%%%%%%%%%%%%%%%%%

%\begin{comment}
\section{Acknowledgments}

\noindent
I would like to thank Joachim Schrod and Klaus Guntermann for their
inspiring discussions in early phases of this project. Further, I
would like to thank Gabor Herr, who was as excellent adviser in many
implementation questions especially concerning \Lisp. Without him this
project would not be in its current state. I would like to thank
Prof.\ Waldschmidt who made it possible to finish this project.
Finally, would like to thank the participants in our discussions on
the \xindy mailing list most notably Chris Rowley.
%\end{comment}

\bibliography{bibliographie}
\bibliographystyle{abbrv}

\end{document}


%%%%%%%%%%%%%%%%%%%%%%%%%%%%%%%%%%%%%%%%%%%%%%%%%%%%%%%%%%%%%%%%%%%%%%

$Log$
Revision 1.1  1997/11/25 19:06:08  kehr
Initial checkin of both texts.

