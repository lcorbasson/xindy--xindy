%%
%% $Id$
%%
%% Document:
%%


%%\documentclass[11pt]{article}
\documentclass[11pt]{report}

\usepackage{oldgerm}
\usepackage{a4}
\usepackage{xspace}
\usepackage{afterpage}
\usepackage{theorem}
\usepackage{fancyheadings}
\usepackage{ifthen}
%%\usepackage{shortvrb}
\usepackage[hang,bf,small]{caption}

%\usepackage{multicol}

\usepackage{epsfig}

% its absolutely necessary to include german AFTER epsfig
\usepackage{german}
\usepackage[isolatin]{inputenc}
%\usepackage{isolatin1}


%% These are my styles

%% Need to set \ReportMode=1 because of the incompatibilities with
%% foil.cls. See stdwrk.sty for further details.

\newcommand{\ReportMode}{1}
\usepackage{stdwrk}

\usepackage{xindy}
\usepackage{figsect}
\usepackage{environments}
\usepackage[help,level4]{tableofcontents}
\usepackage[help,ParagraphLikeSections]{sectioning}
\usepackage{footnotelist}


%%%%% Here comes stuff from Guy Steele..

%%\makeatletter
%%% The \null in the following is intended to suppress hyphenation
%%% in code words not already containing hyphens.  The \leavevmode
%%% is needed to prevent vertical mode from swallowing the \null.
%%%%%\def\cd#1{\leavevmode{\cf \null#1}}
%%\def\cd{\leavevmode\begingroup\cf\@cd}
%%\def\@cd#1{\null#1\endgroup}
%%\def\cf{\ttfamily\frenchspacing}
%%\makeatother

%%%%% stuff from Guy Steele ends here.


%\makeindex

%%
%% $Id$
%%
%% Document: Praeambel mit Definitionen
%%

\hyphenation{%
  Lo-ka-tions-re-fe-renz
  Lo-ka-tions-re-fe-ren-zen
  In-de-xes In-de-xen
  Do-ku-ment-alpha-bet
  Do-ku-ment-alpha-bets
  Do-ku-ment-alpha-be-tes
  Sor-tier-alpha-bet
  Sor-tier-alpha-bets
  Sor-tier-alpha-bet-es
  Sor-tier-ungs-alpha-bet
  Sor-tier-ungs-alpha-bets
  Sor-tier-ungs-alpha-be-tes
  Kon-fi-gu-ra-tion
  Ba-sis-alpha-bet
  Ba-sis-alpha-bets
  Ba-sis-alpha-be-tes
  Sym-bol-alpha-bet
  Sym-bol-alpha-bets
  Sym-bol-alpha-be-tes
  Ka-te-go-rie-attri-but
  Ka-te-go-rie-attri-buts
}

%% Local Variables:
%% mode: latex
%% TeX-master: "makeindex4.tex"
%% End:

%%
%% $Log$
%% Revision 1.3  1995/11/03 15:53:17  kehr
%% Neuformulierung der join-range()-Funktion.
%%
%% Revision 1.2  1995/10/20  11:57:33  kehr
%% Korrektur nach Klaus' Durchsicht.
%%
%% Revision 1.1  1995/10/16  17:31:52  kehr
%% Initial checkin of Report and Presentation.
%%
%% Revision 1.21  1995/10/06  23:05:18  kehr
%% Korrektur nach der Durchsicht von Karin.
%%
%% Revision 1.20  1995/09/22  01:12:08  kehr
%% Zweite �berarbeitung nch der inhaltlichen Korrektur. Au�erdem habe
%% ich das Logo zu MacIndex ver�ndert. Hat jetzt mehr pepp !
%%
%% Revision 1.19  1995/09/21  00:05:48  kehr
%% Erste Ver�nderungen nach der inhaltlichen Korrektur durch Joachim am
%% 20.Sep.95. Fast alle Dateien d'sind davon betroffen. Au�erdem sind noch zwei
%% neue Abbildungen hinzugekommen.
%%
%% Revision 1.18  1995/09/06  18:52:52  kehr
%% Made final changes before giving for correction.
%%
%% Revision 1.17  1995/08/28  18:08:19  kehr
%% Neue Einspielung der xfig-Dateien
%%
%% Revision 1.16  1995/07/04  09:46:31  kehr
%% Weitere �nderungen. Bin aber fast fertig.
%%
%% Revision 1.15  1995/07/04  00:46:54  kehr
%% Bald ist's soweit ;-)
%% Ich habe heute die generelle Umstrukturierung vorgenommen und einige
%% Teile herausgeschmissen. Die Indexverarbeitung mu� noch �berarbeitet werden.
%%
%% Revision 1.14  1995/06/18  23:32:25  kehr
%% Schlu� f�r heute. Genug geschafft.
%%
%% Revision 1.13  1995/06/17  20:36:31  kehr
%% Habe die Lokationsreferenzverarbeitung umstrukturiert und besser
%% definiert. DIe Buchstabengruppen m�ssen noch beendet werden und der
%% Algorithmus zum Mischen und Sortieren der Lokationsreferenzen mu�
%% fertiggestellt werden.
%%
%% Revision 1.12  1995/06/13  21:55:17  kehr
%% Habe heute die Formulierung des Algorithmus controlled-jojo-traverse
%% fertiggestellt. Desweiteren Fehler in der Anwendung der \lindent-Umgebung
%% gefunden. Ich mu� noch die Matrix f�r die Definition der Ausgabekommandos
%% und der Angabe im Indexstyle entwickeln.
%%
%% Revision 1.11  1995/06/09  20:59:52  kehr
%% Superviel gemacht heute ;-)
%%
%% Revision 1.10  1995/06/08  20:19:48  kehr
%% Bibliographie erweitert.
%%
%% Revision 1.9  1995/06/06  23:50:15  kehr
%% Modellentwurf weitergebracht.
%%
%% Revision 1.8  1995/06/06  11:50:36  kehr
%% Weitere Bearbeitung des Modellentwurfs.
%%
%% Revision 1.7  1995/05/31  16:14:59  kehr
%% Dokumant- und Sortierungsalphabet entwickelt. Makefile�nderungen und
%% Stylever�nderungen.
%%
%% Revision 1.6  1995/05/28  21:37:12  kehr
%% Neue �berarbeitete Version.
%% Inhaltliche �nderungen:
%%   Glossar hinzugenommen. Einleitung mit Datenflu�graph. Kleinere
%%   �nderungen an der Beschreibung des International Makeindex.
%% System�nderungen:
%%   Makefile-�nderungen, Stil�nderungen, Titelseite
%%
%% Revision 1.5  1995/05/05  23:07:15  kehr
%% Ge�nderte Datenstrukturen mit enumerate und neuen labels f�r enumerate
%%
%% Revision 1.4  1995/05/05  22:25:06  kehr
%% Ge�nderte Struktur mit einleitung.tex
%% Zwischenspeicherung vor der Umstellung der Definitionen
%%
%% Revision 1.3  1995/04/30  16:14:11  kehr
%% Trennung in Einleitung, Einf�hrung und Analyse. Evtl. sollten die Filenamen
%% entsprechend anepa�t werden. Dar�berhinaus Analyse weitergef�hrt.
%%
%% Revision 1.2  1995/04/25  01:09:40  kehr
%% Analyse und Modellentwurf weitergebracht.
%%
%% Revision 1.1  1995/04/22  21:05:41  kehr
%% Erstes Setup der Studienarbeit des Makeindex4-Projektes
%%

%%% Local Variables:
%%% mode: plain-tex
%%% TeX-master: t
%%% End:


\tolerance=1000
\emergencystretch=20pt
%\parskip=6pt

\overfullrule 2mm

%
% aktivieren der fancyheadings
%
\pagestyle{fancy}
\newcommand{\titleshape}[1]{\textbf{#1}}
%
% falls die Headings �ber die Marginalien hinausragen sollen
%
%\addtolength{\headwidth}{\marginparsep}
%\addtolength{\headwidth}{\marginparwidth}
%
% Korrektur um overfull vbox zu unterbinden
%
\addtolength{\headheight}{2pt}
%
% section und subsection im Kopf redefinieren
%
\renewcommand{\chaptermark}[1]{\markright{\textbf{\thechapter}\ %
      \titleshape{#1}}}
\renewcommand{\sectionmark}[1]{\markright{\textbf{\thesection}\ %
      \titleshape{#1}}}
%
% linken und rechten Kopf definieren
%
\lhead{\fancyplain{}{\let\uppercase\relax\bfseries\rightmark}}
\rhead{\fancyplain{}{\bfseries\thepage}}
\cfoot{}

%% Aktivieren des Pipe-Zeichens als Abbrev f�r \verb|...|
%% Dies mu� an den gew�nschten Stellen im Text geschehen, da eine
%% globale Definiton nicht m�glich ist.
%\MakeShortVerb{\|}
%\DeleteShortVerb{\|}

\begin{document}

%% Local Variables:
%% mode: latex
%% TeX-master: "makeindex4.tex"
%% End:

%%
%% $Log$
%% Revision 1.4  1995/11/15 14:58:13  kehr
%% Final correction (I hope so).
%%
%% Revision 1.3  1995/11/14  16:05:58  kehr
%% Made two more corrections on the report.
%%
%% Revision 1.2  1995/11/08  16:17:04  kehr
%% New correction.
%%
%% Revision 1.1  1995/10/16  17:31:56  kehr
%% Initial checkin of Report and Presentation.
%%
%% Revision 1.15  1995/10/06  23:05:18  kehr
%% Korrektur nach der Durchsicht von Karin.
%%
%% Revision 1.14  1995/09/22  01:12:09  kehr
%% Zweite �berarbeitung nch der inhaltlichen Korrektur. Au�erdem habe
%% ich das Logo zu MacIndex ver�ndert. Hat jetzt mehr pepp !
%%
%% Revision 1.13  1995/09/21  00:05:48  kehr
%% Erste Ver�nderungen nach der inhaltlichen Korrektur durch Joachim am
%% 20.Sep.95. Fast alle Dateien d'sind davon betroffen. Au�erdem sind noch zwei
%% neue Abbildungen hinzugekommen.
%%
%% Revision 1.12  1995/09/06  18:52:53  kehr
%% Made final changes before giving for correction.
%%
%% Revision 1.11  1995/07/04  00:46:55  kehr
%% Bald ist's soweit ;-)
%% Ich habe heute die generelle Umstrukturierung vorgenommen und einige
%% Teile herausgeschmissen. Die Indexverarbeitung mu� noch �berarbeitet werden.
%%
%% Revision 1.10  1995/06/17  20:36:31  kehr
%% Habe die Lokationsreferenzverarbeitung umstrukturiert und besser
%% definiert. DIe Buchstabengruppen m�ssen noch beendet werden und der
%% Algorithmus zum Mischen und Sortieren der Lokationsreferenzen mu�
%% fertiggestellt werden.
%%
%% Revision 1.9  1995/06/09  20:59:53  kehr
%% Superviel gemacht heute ;-)
%%
%% Revision 1.8  1995/06/08  00:35:55  kehr
%% Was soll ich blo� schreiben ???
%%
%% Revision 1.7  1995/06/07  20:59:14  kehr
%% Und weiter am Modellentwurf. Spezifikation des Indexstyles vorerst
%% fertig. Es fehlt noch die Eingabegrammatik.
%%
%% Revision 1.6  1995/06/06  17:51:03  kehr
%% Commit um die �nderungen festzuhalten.
%%
%% Revision 1.5  1995/05/29  00:22:44  kehr
%% Die Einleitung ist somweit fertig und die Analyse mu� jetzt noch
%% beendet werden. Mir fehlt da noch eine vern�nftige Idde f�r die
%% Alphabete und deren Definitionen.
%%
%% Revision 1.4  1995/05/28  21:37:12  kehr
%% Neue �berarbeitete Version.
%% Inhaltliche �nderungen:
%%   Glossar hinzugenommen. Einleitung mit Datenflu�graph. Kleinere
%%   �nderungen an der Beschreibung des International Makeindex.
%% System�nderungen:
%%   Makefile-�nderungen, Stil�nderungen, Titelseite
%%
%% Revision 1.3  1995/05/05  22:25:06  kehr
%% Ge�nderte Struktur mit einleitung.tex
%% Zwischenspeicherung vor der Umstellung der Definitionen
%%
%% Revision 1.2  1995/04/25  01:09:41  kehr
%% Analyse und Modellentwurf weitergebracht.
%%
%% Revision 1.1  1995/04/22  21:05:41  kehr
%% Erstes Setup der Studienarbeit des Makeindex4-Projektes
%%
