% $Id$
%----------------------------------------------------------------------
%
% Context-Based Markup
% Technical Report 1997
% Institut f�r Theoretische Informatik, TH Darmstadt
%
% [LaTeX2e]
% (history at end)

%\documentclass[titlepage,11pt]{article}
\documentclass[titlepage,11pt,draft]{article}

\usepackage{jsdoc}
\usepackage{a4-9}
\usepackage{xspace}
%\usepackage{itititle}
\usepackage{environments}
%\usepackage{ifthen}
%\usepackage{theorem}
\usepackage{bslash}
\usepackage{alltt}
\usepackage{enumerate}
\usepackage{verbatim}
%\usepackage{afterpage}
\usepackage{rotating}
\usepackage{epsfig}
\usepackage[hang,sf]{caption}
%\usepackage[textures,bottomafter]{draftcopy}
%\usepackage{version}

\def\TechReport{1}
\def\ReportMode{0}
\usepackage{stdwrk}

\usepackage{xindy}

\makeatletter
\renewcommand{\@oddfoot}{\textsf{DRAFT}\hfil\textsf{\thepage}\hfil\textsf{DRAFT}}
\makeatother

\frenchspacing

%%\def\mkidx{{\texttt{makeindex}}\xspace}
\def\mkidx{\emph{MakeIndex}\xspace}
\def\Xindy{{\normalfont\textsf{xindy}}\xspace}
\def\imkidx{{\emph{International MakeIndex}}\xspace}

\newcommand{\xindy}{%%
   \mbox{\normalfont%%
     \textsf{x\kern-0.6pt%%
       \shortstack{{\scriptsize$\circ$}\\[-2pt]\i}%%
       \kern-1pt%%
       ndy}%%}%%
     }\xspace}

\newcommand{\XINDY}{%
   \mbox{\normalfont\huge%%
         \textsf{x\shortstack{{\large$\circ$}\\\i}ndy}}}

%\renewcommand{\thefootnote}{\fnsymbol{footnote}}

\newlength\vertspace

\newenvironment{mfigure}[2]{%
  \begin{figure}[htbp]%
    \renewcommand{\tfigurecaption}%
                 {\caption{#1}}        % save the caption
    \renewcommand{\tfigurelabel}%
                 {\label{#2}}          % save the label
    \Hrule                             % start with horizontal line
    \vspace*{3mm}                      % small vertical space
}
{\vspace*{3mm}                         % another skip at the end
  \Hrule                               % another line
  \normalfont                          %
  \vskip\vertspace                     %
  \tfigurecaption                      % insert caption
  \tfigurelabel                        % and label
\end{figure}%                          % ok, we're done
}

\newenvironment{mtable}[2]{%
  \begin{table}[htbp]%
    \renewcommand{\ttablecaption}%
                 {\caption{#1}}        % save the caption
    \renewcommand{\ttablelabel}%
                 {\label{#2}}          % save the label
    \Hrule                             % start with horizontal line
    \vspace*{3mm}                      % small vertical space
}
{\vspace*{3mm}                         % another skip at the end
  \Hrule                               % another line
  \normalfont                          %
  \vskip\vertspace                     %
  \ttablecaption                       % insert caption
  \ttablelabel                         % and label
\end{table}%                           % ok, we're done
}

\begin{document}

\date{April 1997}

\title{A Simple Context-Based Markup Algorithm and its Efficient
  Implementation in \textsc{Clos}\\[2ex]
  \Huge\sffamily %%DRAFT VERSION
  }

\author{Roger Kehr}

%\iti{Bericht TI-XXX/97}

\maketitle

\def\emdash{--}

\def\Lisp{{\normalfont\textsc{Lisp}}\xspace}
\def\CL{{\normalfont\textsc{Common Lisp}}\xspace}
\def\CLOS{{\normalfont\textsc{Clos}}\xspace}
\def\COST{{\normalfont\textsc{CoST}}\xspace}
\def\STIL{{\normalfont\textsc{Stil}}\xspace}
\def\dps{document preparation system\xspace}
\def\nroff{{\tt nroff}\xspace}
\def\term#1{\emph{#1}}
\def\pair#1{\mbox{$\langle$}#1\mbox{$\rangle$}}
\def\ts{\hspace*{0.9ex}}

%%%%%%%%%%%%%%%%%%%%%%%%%%%%%%%%%%%%%%%%%%%%%%%%%%%%%%%%%%%%

{
  \let\oldbfseries\bfseries
  \def\newbfseries{\sffamily\oldbfseries}
  \let\bfseries\sffamily

  \begin{abstract}
    This report describes a simple though powerful context-based
    markup algorithm and its implementation.

    Markup algorithms are often used to tag data structures with
    markup for representation purposes. They often follow an
    event-dispatching mechanism that allows the declaration of markup
    tags depending on the context in which an event is occurs.

    We describe the markup algorithm which is used in the \xindy index
    processor and show how the multi-method dispatching facilities of
    \CLOS could be easily exploited to obtain an efficient
    implementation.
  \end{abstract}
}

\parskip=0.2\baselineskip plus 2pt minus 1pt


%%%%%%%%%%%%%%%%%%%%%%%%%%%%%%%%%%%%%%%%%%%%%%%%%%%%%%%%%%%%

\newpage

\section{Introduction}

It is often necessary to output some data structure stored in main
memory into a data stream using some particular form of
\term{representation}.  Somehow this process can be seen as the
reverse work of a \term{parser}. The reasons for representing data
structures in a stream are often \term{persistence}, the need to store
the data structures for later reuse, and \term{structured markup},
making the representation further available for other post-processing
steps such as typesetting.

We are interested in the problem of \term{tagging} the data to obtain
a specific markup needed for typesetting purposes. This problem is
easily solvable as long as the way how markup is assigned to data
objects is static. In this case the markup process can be directly
coded into some algorithm.

If there is a need for \term{configurable markup}, the situation is
more complicated. For an illustration of the difference imagine a tree
consisting of nodes representing different objects that have to be
output.  Usually one traverses the tree and emits appropriate markup
tags when moving from one node to another. This kind of markup is
called \term{environment-based markup}, since the objects (in our case
the nodes of the tree) are encapsulated into an environment of markup
tags.

In almost all cases the nodes of a particular type are tagged
uniformly in the same manner. This kind of markup is often used to
\term{dump} a representation of a tree into a data stream. However,
there are application domains for which one wants to make the issued
markup tags depending to the node's position within the tree.

Therefore, we can roughly divide the markup schemes into
\term{context-free} and \term{context-based} (or
\term{context-sensitive}) markup. In this paper we will further
concentrate on context-based markup.

The rest of this paper is organised as follows. In
section~\ref{relwrk} we first discuss existing systems that already
contributed solutions to this kind of problem. In section~\ref{CCBM}
we describe the application domain for which we needed an efficient
solution to the problem. We present our traversing algorithm and its
efficient implementation in \CLOS~\cite{Keene:88}.



\section{Related Work}
\label{relwrk}

The problem of context-based markup has several interesting general
solutions. We briefly present the solutions of two different systems:
\COST~\cite{CoST} and \STIL~\cite{stil}. Both systems are intended as
back-ends for SGML \term{parsers}. SGML documents must conform to a
structure that is described in a BNF-like grammar, called the
\term{document type definition} (DTD). A document conforming to some
DTD can be represented as a tree having SGML \term{elements} as its
nodes. A SGML parser validates a document against its DTD and produces
a structural description of its contents, the so-called ESIS (Element
Structure Information Set).

Both systems read the output of a SGML-parser and construct an
internal representation of the tree. This tree is then traversed in
some order (typically depth first order) and appropriate markup is
output when entering or leaving a node. The \term{context} of a node
is different in both systems. In \COST all nodes along the path from
the root node to the current node form a node's context. In \STIL the
whole tree forms a node's context.

Both systems allow to specify markup depending on the particular
context in a very powerful way. Each SGML element is represented as
the instance of a particular class (\COST uses [incr
Tcl]~\cite{McLennan:1995:NIT}, \STIL uses \CLOS~\cite{Keene:88} as
their object systems). When a node is entered or left an event is
generated. It is essentially a message sent to the current object
which must offer an appropriate callback method.

The user of these systems simply binds methods to the element classes
that output the desired markup. Both systems offer a rich set of
primitives to make the context information available inside these
methods.

Thus, one can implement very powerful context-based markup strategies
in a very declarative manner.



\section{A Configurable Context-Based Markup Scheme}
\label{CCBM}

\subsection{The Problem}

When implementing the \xindy index processor~\cite{xindy:tr} we were
faced with a similar though less complicated problem. After processing
the index its internal tree representation needs to be output
according to an user definable markup.

The nodes of the tree are instances of internally defined classes as
well as user-defined classes. An example of such user-defined classes
are the so-called \term{location classes} that can be specified in the
\term{index style}\footnote{The index style contains a description of
  the occuring location classes and other objects}. Other nodes in the
tree are lists containing other objects as their elements. Since the
markup of different location classes may be different, we decided to
offer to the users a context-based markup scheme to be sufficient
powerful.

For the intended application domain the functionality of systems such
as \textsc{Stil} was not appropriate and we aimed for a more efficient
solution to the markup problem.\footnote{Especially writing \CLOS
  methods cannot be expected by any non-programmer user.} As a first
restriction we decided to make available only a subset of the elements
along the path as a node's context. Furthermore we observed that most
often there is a need for making the attributes of an instance
accessible as context information.

After these simplification steps we ended up in a very simple model.
The context information for a particular node did not exceed four
different objects. The basic traversing order was the depth first
order and events were generated when entering a node and when leaving
a node. For list nodes we additionally generated an event when moving
from one element in the list to its successor.


\subsection{An Efficient Implementation in CLOS}

The implemented markup system consists of two different parts, (1) the
markup algorithm traversing the data structure and raising appropriate
events, and (2) the user interface, establishing the corresponding
event bindings.

We continue with a description of the underlying context-based markup
scheme and a description of the dispatching mechanism which is
directly implemented using \CLOS method calls. Finally we describe the
user interface responsible for the dynamic establishing of event
bindings.



\subsubsection*{Context Based Markup via Multi-Method Dispatch}
\label{sec:tagging}

The tagging process directly operates on the tree representation as
illustrated in figure~\ref{fig:index-tree}.
%%
\begin{sidewaysfigure}
  \begin{center}
    \input{index-tree.latex}
  \end{center}
  \caption{Skeleton tree structure of an index}
  \label{fig:index-tree}
\end{sidewaysfigure}
%%
For each inner node there exists a path from the root to the current
node. Starting from the root in figure~\ref{fig:index-tree}, such a
path could look like
%%
\def\emrm#1{\mbox{\normalfont\emph{#1}}}%%
{\small\begin{alltt}
       \emrm{index}
         \emrm{letter group =} "G"
           \emrm{index entry =} "foo"
             \emrm{location class =} "page-numbers"
               \emrm{attribute group =} 1
                 \emrm{location reference attribute =} "definition"
\end{alltt}}
%%
\noindent The context of an object consists of this path and all
attributes the object owns. In our system, an event is generated each
time a node is entered for the first time or left forever. The event
contains a certain subset of the information present in the current
context and passes this information to the dispatching unit.

In contrast to \STIL, which represents events as \CLOS method calls
using a \term{single-argument dispatch}, we use in \xindy{} the
\term{multi-method dispatch facilities} of \CLOS to implement the
event bindings. The current context of a node is entirely passed as
the method's arguments to the multi-method dispatcher of \CLOS that
finds the appropriate method.

The method definitions make heavy use of the
\texttt{eql}-specialisation feature of method arguments. This feature
specialises a method not only on a certain class, which is usually the
case in traditional object-oriented languages, but rather specialises
on a single instance which can be determined at run-time. This makes
it possible to specialise an event-binding to only a particular subset
of the context arguments.

A sample method that is called in response to an event generated when
entering a location reference owning the attribute \emph{definition},
belonging to class \emph{pagenums}, and being positioned on an
arbitrary depth is represented through the following method.

{\small\begin{verbatim}
   (defmethod do-markup-locref-open
       ((attr      (eql #<attribute "definition">))
        (locrefcls (eql #<class     "pagenums">))
        (depth     number))
     (do-markup-string "\macro{"))
\end{verbatim}
    }

\noindent This method simple outputs the string ``\verb|\macro{|'' to the markup  stream. It is specialised\footnote{Specialisation is done with the
    argument to the \texttt{eql}-specialiser. The notation
    \texttt{\#<...>} simply denotes an instance of some class.} to an
  attribute instance and a location reference class
  instance\footnote{Location classes are themselves instances of
    another class.}. The above method is specialised to only two of
  three arguments. Since the argument \texttt{depth} is not
  specialised any further this method matches all location references
  of the given class and attribute appearing at any depth.


\subsubsection*{The User Interface}

The user interface is primarily responsible to establish appropriate
event-bindings. Since event bindings are implemented as \CLOS methods
the user interface consists of \Lisp macros expanding to appropriate
method definitions. Defining methods at run-time is a very powerful
property of \CLOS on which this approach relies mostly.

All markup commands in index style have the following form:
%%
\def\itarg#1{\mbox{\normalfont\emph{#1}}}
\def\rmarg#1{\mbox{\normalfont #1}}
{\small\begin{alltt}
   (markup-\itarg{index-tree-node-name}
       \rmarg{[}:open \itarg{markup}\rmarg{]} \rmarg{[}:close \itarg{markup}\rmarg{]} \rmarg{[}:sep \itarg{markup}\rmarg{]}\footnote{Option \texttt{:sep} is applicable for list nodes only.}
       \rmarg{[\emph{context-dependent options such as} \texttt{:attr}, \texttt{:group}, \texttt{:class}, \texttt{:depth}, \ldots]})
\end{alltt}}

\noindent Since all fields are optional one can assign
the markup to only a subset of the context arguments. The accessible
context arguments depend on the type of the object that is to be
output. We have not made the whole context available to the user, but
instead have selected a meaningful subset for each markup command.

\newpage

The \term{context-based markup specification} allows to assign a
generic markup, for instance, for all location reference classes with
the command
{\small%%
\begin{verbatim}
   (markup-locref :open  "\generic{"
                  :close "}")
\end{verbatim}
}%%
\noindent but redefining the markup for the instances of a certain
location class as follows:
{\small%%
\begin{verbatim}
   (markup-locref :class "special-class"
                  :open  "\special{"
                  :close "}")
\end{verbatim}
}%%

\noindent The event dispatcher has to decide which of the markup schemes
is the most specialised one that matches the given arguments.  In the
above situation the second markup is selected for a location reference
of class \texttt{special-class}, but for all others the generic markup
is used, as defined by the first command.

%\enlargethispage{\baselineskip}

As an example of the run-time definition of methods consider the
following user interface declaration, which is specialised on all
three possible arguments,
%%
{\small\begin{verbatim}
   (markup-locref :open "\defin{" :close "}"
                  :attr "definition"
                  :class "pagenums"
                  :depth 1 ))
\end{verbatim}
    }
%%
\noindent roughly expands to
%%
{\small%%
\begin{verbatim}
   (progn
     (defmethod do-markup-locref-open
         ((attr      (eql #<attribute "definition">))
          (locrefcls (eql #<class     "pagenums">))
          (depth     (eql 1)))
       (do-markup-string "\defin{"))

     (defmethod do-markup-locref-close
         ((attr      (eql #<attribute "definition">))
          (locrefcls (eql #<class     "pagenums">))
          (depth     (eql 1)))
       (do-markup-string "}")))
\end{verbatim}
  }%%
\noindent Thus, we use \texttt{eql}-specialisation on instances for
all three specifiers. Both methods simply output the open (resp.\
close) tag by calling the function \texttt{do-markup-string}.

A question that comes up when analysing this implementation is that of
the \term{metrics} that are used to find the \term{most-specialised}
method of a certain set of methods. The problem arises when
considering an event with the arguments \emph{attr=definition},
\emph{locrefcls=pagenums} that must dispatch to exactly one of the two
following methods:

{\small%%
\begin{verbatim}
   (defmethod do-markup-locref-open
      ((attr      (eql #<attribute "definition">))
       (locrefcls class) ... ))

   (defmethod do-markup-locref-close
      ((attr      attribute)
       (locrefcls (eql #<class "pagenums">)) ... ))
\end{verbatim}
  }

\noindent Both methods match the given event but since exactly one method must
be called the dispatcher must somehow decide what is the most
specialised method of both. Since the method dispatcher of \CLOS has a
built-in metric that always finds one most-specialised method we
simply use it to resolve this kind of ambiguity for us.\footnote{In
  \CLOS the first method would be selected since the first argument
  \emph{attribute} is more special than the second attribute. We have
  chosen the positions of the arguments in a way that seems to be most
  useful in practice.}

\bigskip

\noindent\textsf{Example 1: Tagging of Ranges}

\smallskip

\noindent A common way of tagging ranges is as follows: a range of
length 1 is printed with the starting page number and the suffix `f.',
those of length 2 with suffix `f\kern0ptf.', and all others in the form
`$x$--$y$'. Assume we want to do this for the location class
\emph{pagenums} we can specify the markup as follows:

{\small\begin{verbatim}
  (markup-range :class "pagenums" :close "f."  :length 1 :ignore-end)
  (markup-range :class "pagenums" :close "ff." :length 2 :ignore-end)
  (markup-range :class "pagenums" :sep "--")
\end{verbatim}
    }

\noindent The first two commands specialise on two different depths',
whereas the third command can be seen as the otherwise-case that is
invoked for all ranges which are not of length one or two. The special
switch \texttt{:ignore-end} causes the second location reference to be
suppressed in the resulting output.

\bigskip

\noindent\textsf{Example 1: Tagging of Hierarchies}

\smallskip

\noindent Sometimes references to appendices are tagged in the following way:
\begin{center}
  \itshape A-1, A-7, A-11, B-3, B-4, B-5, C-1, C-8, C-12, C-13, C-22
\end{center}
\noindent This kind of markup can be achieved with the following
markup commands:

{\small\begin{verbatim}
   (markup-locref-list       :class "appendices" :sep ", ")
   (markup-locref-layer-list :class "appendices" :sep "-")
\end{verbatim}
}
\noindent The first command indicates that all location references shall be
separated by a comma followed by a blank character. The second line
says that the different layers the location references consist of
shall be separated by a hyphen character.\footnote{The location
  reference \emph{A-1} consists of the following list of layers
  (\emph{A}\ \emph{1}).} Our example could also be represented using
a much more compact form:
\begin{center}
   \itshape A 1, 7, 11; B 3, 4, 5; C 1, 8, 12, 13, 22
\end{center}
\noindent To achieve this markup one needs to transform the location
references into a hierarchy. This yields the location reference
\emph{A} and its sub-references \emph{1}, \emph{7}, and
\emph{11}. The following markup can then be used to obtain the
desired result:

{\small\begin{verbatim}
   (markup-locref-list :class "appendices" :depth 0           :sep "; ")
   (markup-locref-list :class "appendices" :depth 1 :open " " :sep ", ")
\end{verbatim}
}
\noindent The locations at depth zero are separated by a semicolon whereas
the ones at depth one are separated by a colon.


\section{Conclusion}

We have presented a very efficient implementation of a simple
context-based markup algorithm in \CLOS. The implementation mostly
benefits from three interesting properties of \CLOS, (a) the ability
to define methods at run-time using the macro expansion mechanism, (b)
the \texttt{eql}-specialisation feature that allows specialisation of
methods down to single instances of a class, and (c) the multi-method
dispatching paradigm. These features made it possible to implement the
markup algorithm with minimal effort. The resulting code is extremely
compact and easily maintainable and extensible.



%%%%%%%%%%%%%%%%%%%%%%%%%%%%%%%%%%%%%%%%%%%%%%%%%%%%%%%%%%%%

%\newpage

\bibliographyIntroduction={The following books and papers were
  referenced in this report.}

\bibliography{bibliographie}
\bibliographystyle{abbrv}


\end{document}


%%%%%%%%%%%%%%%%%%%%%%%%%%%%%%%%%%%%%%%%%%%%%%%%%%%%%%%%%%%%%%%%%%%%%%
%
% $Log$
% Revision 1.1  1997/03/28 13:01:27  kehr
% Initial Checkin of the Context-Based Markup Report.
%
%
