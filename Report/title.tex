%%
%% $Id$
%%
%% Document: Titelseite
%%

\begin{titlepage}

\vspace*{\stretch{1}}
\HRule

\begin{flushright}
\vspace*{1.5mm}

{\MKXVIER%%
\\[5mm]
}

%%{\LARGE$\EuFrak{McIndex}$}
%%{\LARGE\textfrak{McIndex}}
%%{\LARGE\textswab{McIndex}}
%%{\LARGE\textbf{M\raisebox{0.5ex}{\normalsize c}Index}}
%%{\LARGE{M\raisebox{0.5ex}{\small c}Index}}
%%{\LARGE\textsc{M\raisebox{0.7ex}{\large c}Index}}
%%{\LARGE\textsc{M\raisebox{0.7ex}{\large ac}Index}}




{\rmfamily\bfseries\Large
Ein System zur Indexgenerierung}

\end{flushright}

\HRule
\vspace*{\stretch{1}}

\begin{flushright}

\large
Studienarbeit
\\[2mm]
{\scshape Roger Kehr}
\\[2mm]
Oktober 1995
\\[8mm]
% kludge to get athene-logo printed (displays wrong)
\font\athenefont=athenes scaled 800
\def\athene{{\athenefont\char0}}
\parbox{3cm}{\athene}
\parbox[c]{7cm}{%
  \raggedleft
  Institut f�r Theoretische Informatik
  \\[2mm]
  FG Systemprogrammierung
}
\end{flushright}

\vspace*{\stretch{0.5}}

\end{titlepage}

%% Local Variables:
%% mode: latex
%% TeX-master: "makeindex4.tex"
%% End:

%%
%% $Log$
%% Revision 1.1  1995/10/16 17:31:57  kehr
%% Initial checkin of Report and Presentation.
%%
%% Revision 1.11  1995/10/06  23:05:19  kehr
%% Korrektur nach der Durchsicht von Karin.
%%
%% Revision 1.10  1995/09/22  01:12:09  kehr
%% Zweite �berarbeitung nch der inhaltlichen Korrektur. Au�erdem habe
%% ich das Logo zu MacIndex ver�ndert. Hat jetzt mehr pepp !
%%
%% Revision 1.9  1995/09/21  00:05:49  kehr
%% Erste Ver�nderungen nach der inhaltlichen Korrektur durch Joachim am
%% 20.Sep.95. Fast alle Dateien d'sind davon betroffen. Au�erdem sind noch zwei
%% neue Abbildungen hinzugekommen.
%%
%% Revision 1.8  1995/08/28  18:08:19  kehr
%% Neue Einspielung der xfig-Dateien
%%
%% Revision 1.7  1995/07/04  00:46:56  kehr
%% Bald ist's soweit ;-)
%% Ich habe heute die generelle Umstrukturierung vorgenommen und einige
%% Teile herausgeschmissen. Die Indexverarbeitung mu� noch �berarbeitet werden.
%%
%% Revision 1.6  1995/06/13  21:55:18  kehr
%% Habe heute die Formulierung des Algorithmus controlled-jojo-traverse
%% fertiggestellt. Desweiteren Fehler in der Anwendung der \lindent-Umgebung
%% gefunden. Ich mu� noch die Matrix f�r die Definition der Ausgabekommandos
%% und der Angabe im Indexstyle entwickeln.
%%
%% Revision 1.5  1995/05/31  19:18:54  kehr
%% Fertigstellung des Analyse-Abschnitts (Hoffentlich ;-).
%%
%% Revision 1.4  1995/05/28  21:37:13  kehr
%% Neue �berarbeitete Version.
%% Inhaltliche �nderungen:
%%   Glossar hinzugenommen. Einleitung mit Datenflu�graph. Kleinere
%%   �nderungen an der Beschreibung des International Makeindex.
%% System�nderungen:
%%   Makefile-�nderungen, Stil�nderungen, Titelseite
%%
%% Revision 1.3  1995/04/30  16:14:12  kehr
%% Trennung in Einleitung, Einf�hrung und Analyse. Evtl. sollten die Filenamen
%% entsprechend anepa�t werden. Dar�berhinaus Analyse weitergef�hrt.
%%
%% Revision 1.2  1995/04/25  01:09:41  kehr
%% Analyse und Modellentwurf weitergebracht.
%%
%%
