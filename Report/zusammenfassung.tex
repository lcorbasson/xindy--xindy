%%
%% $Id$
%%
%% Document: Zusammenfassung des `makeindex4' - Projekts
%%



\chapter{Zusammenfassung}
\label{sec:zusammenfassung}

Das \mkxvier-System ist eine Weiterentwicklung der \textsf{makeindex}-
und \textit{International MakeIndex}-Systeme.
%%Dies soll durch die Namensgebung der Zahl vier im Index dargestellt
%%werden, da die Vorg�ngerversionen unter den Bezeichnungen 2.x und 3.x
%%verf�gbar sind.

Durch eine komplette theoretische Neuanalyse von Indexen konnten die
Erfordernisse an ein Indexsystem �berarbeitet werden. Die intensive
Analyse von Lokationsklassen und deren Verarbeitungsprozesse sowie die
neuartige universelle Ausgabeformatierung bilden den Hauptteil dieser
Studienarbeit.

%Daraus entstand nach einer datengerechten Modellierung das vorliegende
%System, welches auch dem Bedarf nach h�chstm�glicher
%Konfigurierbarkeit gerecht wird. Es stellt dem Benutzer ein
%leistungsf�higes Werkzeug zur Indexerstellung zur Verf�gung, dessen
%Grundstruktur m�glichst wenig Einschr�nkungen unterliegt.

Die Implementierungszeit konnte durch die vollst�ndige theoretische
Erarbeitung des Modells und der Verarbeitungsverfahren drastisch
verk�rzt werden.

W�hrend der Modellierung f�hrte vor allen Dingen die semantische
Bedeutung, die bestimmten Ausgabeformen von Lokationsreferenzen
zugeordnet werden mu�te, zu Problemen. Besondere Schwierigkeit lag in
der Kategorisierung von Ausgabeformen und inwieweit
Lokationsreferenzen unterdr�ckbar sein sollten. Wir haben uns hier zum
einen an existierende Indexe gehalten und zum anderen auch eigene
�berlegungen angestellt, die in die Modellierung eingeflossen sind.
Um die Komplexit�t niedrig zu halten und auch den Benutzer nicht mit
�bertriebener Konfigurierbarkeit zu �berfordern, haben wir uns
letztendlich f�r pragmatische, aber leistungsf�hige L�sungen
entschieden.

Die Erstellung eines Indexes ist generell auch eine Geschmacksfrage.
An dieser Stelle ist es aufgrund der Vielfalt der potentiellen W�nsche
der Autoren schwierig, eine optimale Gesamtl�sung auszuarbeiten.


\section*{Weiterf�hrende Arbeiten}
\label{sec:weiterArbeiten}

Im Rahmen der Modellierung eines Indexes und der damit verbundenen
Algorithmen sind zwei Probleme grunds�tzlicher Natur in den
Vordergrund getreten.

Das erste Problem ist die Erkenntnis, da� die Lokationsreferenzen als
Informationstr�ger �ber das zu indizierende Dokument nicht ausreichend
sind, um bestimmte in der Praxis vorkommende Indexe nachzubilden. Das
Hauptproblem ist in fehlendem Dokumentwissen\remark{Dokument\-wissen}
begr�ndet. Bereits die einfachste Gliederungsform in Punktnotation
(\textsf{1.2.3}\ldots) mit verschiedenen Gliederungstiefen macht die
Bestimmung des Nachfolgers einer Lokation unl�sbar, sofern dem System
die ben�tigten Informationen nicht zus"atzlich zugef�hrt werden.

Das zweite Problem besteht darin, da� ein Index sich auch
typographischen Satzregeln unterwerfen mu�, die \zB verlangen, da� die
Ersetzung von Stichw�rtern durch Wiederholungssymbole (\emph{siehe
  Abschnitt \ref{sec:travAusgabebaum}}) an Seiten- und Spaltenanf�ngen
unterdr�ckt werden sollte.\remark{textsatz\-abh�ngige Indexausgabe} Um
dieses Problem zu l�sen, ben�tigt das Textsatzsystem allerdings
Informationen, die w�hrend des Indexierungsvorgangs ermittelt werden.
Wie man sieht, sind diese Probleme nur dann l�sbar, wenn ein
geeigneter Informationsaustausch zwischen beiden Systemen stattfindet.
Die untersuchten Textsatzsysteme bieten allerdings keine direkte
Unterst�tzung, um ein Indexierungssystem direkt an den Textsatzproze�
anzukoppeln.

Interessant ist es die Spezifikation der Ausgabeformatierung und der
Regeln zum Sortieren und Mischen der Lokationsreferenzen zu
"uberarbeiten. Im Moment ist Wissen "uber die zugrundeliegenden
Algorithmen n"otig, um eine solche Spezifikation zu erstellen. Dies
ist jedoch f"ur einen Benutzer eine sehr unbefriedigende L"osung und
eine Untersuchung, ob \emph{interaktive} Werkzeuge dieses Problem
vereinfachen k"onnen, ist mit Sicherheit sinnvoll.

Des weiteren k�nnte es sinnvoll sein, das Konzept der Indexklassen
weiter zu untersuchen. Insbesondere k�nnte man sich mit der Bildung
einer Hierarchie von Klassen besch�ftigen, um Klasseneigenschaften
durch Vererbung\remark{Indexklassen\-vererbung} weiterzugeben. In
diesem Zusammenhang kann man auch die Gesamt- und Masterindexe und
ihre Eigenschaften vertieft untersuchen.


%% Local Variables:
%% mode: latex
%% TeX-master: "makeindex4.tex"
%% End:

%%
%% $Log$
%% Revision 1.4  1995/11/14 16:06:01  kehr
%% Made two more corrections on the report.
%%
%% Revision 1.3  1995/11/08  16:17:14  kehr
%% New correction.
%%
%% Revision 1.2  1995/10/20  11:57:35  kehr
%% Korrektur nach Klaus' Durchsicht.
%%
%% Revision 1.1  1995/10/16  17:32:00  kehr
%% Initial checkin of Report and Presentation.
%%
%%

