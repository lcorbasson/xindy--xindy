%%
%% $Id$
%%
%% Document: Glossar: `makeindex4' - Projekt
%%


\chapter{Glossar}
\label{sec:glossar}

\begin{tfigure}%
  {Index mit Bezeichnung der Komponenten eines Indexeintrags}%
  {fig:glo:1}%
  \centering
  \begin{minipage}{8cm}
    \begin{mkindex}
      \idx B�ume
      \subidx AVL, 3.3
      \subidx nat�rliche, 3.1
      \idx
      \fbox{%
        \fbox{Suche, bin�re}\,\footnote{Stichwort},%
        \fbox{%
          \fbox{2.2}\,\footnote{Strukturreferenz} %
          \fbox{\emph{f.}}\,\footnote{Lokationsattribut}%
          }\,\footnote{Attributierte Strukturreferenz}\,%
        }\,\footnote{Indexeintrag}
      \subidx sequentielle, 2.1
    \end{mkindex}
  \end{minipage}
\end{tfigure}


\begin{description}

\item[Attributierte Strukturreferenz] \mbox{}\\ Strukturreferenz, der
  ein zus�tzliches Attribut zugeordnet ist. Die Lokationsreferenz
  \quasi{35\emph{f.}}, die auf Seite 35 und die folgenden Seiten
  verweist, besteht aus der Strukturreferenz \Strukeins{35} und dem
  Attribut \quasi{\emph{f.}}.

\item[Basisalphabet] \mbox{}\\ Teilmenge eines Dokumentalphabets.
  \emph{Siehe auch Definition~\ref{def:basisalphabet},
    S.\,\pageref{def:basisalphabet}}.


\item[Dokumentalphabet] \mbox{}\\ Gemeinsames zugrundeliegendes
  Alphabet des Textsatzsystems und des Indexsystems. �blicherweise
  ASCII, ISO-Latin--Familie oder Unicode. \emph{Siehe auch
    Definition~\ref{def:dokumentalphabet},
    S.\,\pageref{def:dokumentalphabet}}.

\item[Dokumentwissen] \mbox{}\\ Bezeichnet Informationen, die in der
  Struktur des Dokumentes abgelegt sind und dem Indexierungssystem zur
  korrekten Bildung von Lokationsbereichen mitgeteilt werden m�ssen.
  \emph{Siehe Abschnitt~\ref{sec:nachfolger}}.

\item[Hierarchieebene] \mbox{}\\ Bezeichnet eine einzelne Ebene eines
  hierarchisch untergliederten Stichworts. Das Stichwort
  \Hierzwei{Suche}{bin�re} besteht aus den Hierarchieebenen
  \quasi{Suche} und \quasi{bin�re}.

\item[Index] \mbox{}\\ Bezeichnet ein Stichwortverzeichnis im �blichen
  Sinne. Es besteht aus einer Liste von sortierten Indexeintr�gen.

\item[Indexeintrag] \mbox{}\\ Besteht aus einem Stichwort und einer
  Menge von Lokationsreferenzen. \emph{Siehe auch
    Abbildung~\ref{fig:glo:1}}.

\item[Kategorieattribut] \mbox{}\\ Lokationsreferenzattribut, welches
  zu Kategorisierungszwecken verwendet wird und vom Textsatzsystem zur
  unterschiedlichen optischen Hervorhebung von Lokationsreferenzen
  verwendet werden kann. \emph{Siehe auch
    Abschnitt~\ref{sec:kategorieattribute}}.

\item[Komponententyp] \mbox{}\\ Bezeichnet den Typ einer
  Strukturkomponente.

\item[Lokation] \mbox{}\\ Strukturelles Objekt \bzw Element eines
  Textes. Beispiele f�r solche Lokationen sind \emph{Seiten},
  \emph{Kapitel}, \emph{Abschnitte}, \emph{Unterabschnitte},
  \emph{Anh�nge} \etc{}\,. Jede Lokation kann in einem Dokument nur
  ein einziges Mal auftreten und mu� eindeutig auf"|findbar sein.

\item[Lokationsreferenz] \mbox{}\\ Eindeutiger Bezeichner, der einer
  Lokation zugeordnet ist. Die Lokationsreferenz \quasi{Seite 51}
  verweist auf eindeutige Weise auf bestimmtes Objekt des Dokuments,
  eben die 51.\ Seite.

  %Im \mkxvier{}--System wird eine Lokationsreferenz als
  %\emph{Attributierte Strukturreferenz} behandelt.

\item[Permutierter Index] \mbox{}\\ Enh�lt zu jedem hierarchisch
  gegliederten Stichwort wie \Hierzwei{Suche}{bin�re} auch
  Permutationen der Stichworthierarchien wie \zB
  \Hierzwei{bin�re}{Suche}. \emph{Siehe dazu auch
    Abschnitt~\ref{index:permutierter}}\,.

\item[Referenz] ist gleichbedeutend mit \emph{Lokationsreferenz}.

\item[Sortieralphabet] \mbox{}\\ Definiert ein Alphabet von Worten
  �ber einem Basisalphabet. Ein solches Alphabet bestimmt die
  Sortierordnung innerhalb von Strukturkomponenten einer Lokationsreferenz.
  \emph{Siehe auch Definition~\ref{def:sortierungsalphabet},
    S.\,\pageref{def:sortierungsalphabet}}.

\item[Stichwort] \mbox{}\\ Begriff, auf dessen Vorkommen innerhalb
  eines Textes verwiesen werden soll. Ein strukturierter
  Begriff kann sich aus mehreren Ebenen zusammensetzen. Im
  Beispiel~\ref{fig:glo:1} haben wir die Stichw�rter
  \Hierzwei{B�ume}{AVL}, \Hierzwei{B�ume}{nat�rliche} und
  \Hierzwei{Suche}{bin�re}. Wir verstehen hier als Stichwort alle
  zusammengeh�renden Ebenen eines Begriffs.

\item[Strukturkomponente] \mbox{}\\ Teilkomponente der Struktur einer
  Lokationsreferenz. Die Lokationsreferenz \Strukdrei{3}{.}{1}{.}{2}
  besteht aus den Strukturkomponenten \Strukebene{3}, \Strukebene{1} und
  \Strukebene{2}.

\item[Strukturreferenz] \mbox{}\\ Beschreibt die strukturelle
  Zusammensetzung einer Lokationsreferenz. Die Struktur einer Referenz
  ergibt sich aus der Folge der Strukturkomponenten und der Trennzeichen
  zwischen den Ebenen. Die Lokation \mbox{\quasi{Kapitel--1}} besteht
  aus der 1.\ Ebene \quasi{Kapitel}, dem Trennzeichen \quasi{--} und
  der 2.\ Ebene \quasi{1}. Wir notieren solche Strukturreferenzen als
  \Strukzwei{Kapitel}{--}{1}. Weitere Beispiele sind der Abschnitt
  \Strukdrei{1}{.}{2}{.}{3} oder die Seite \Strukeins{23}.

\end{description}


%% Local Variables:
%% mode: latex
%% TeX-master: "makeindex4.tex"
%% End:

%%
%% $Log$
%% Revision 1.1  1995/10/16 17:31:51  kehr
%% Initial checkin of Report and Presentation.
%%
%% Revision 1.13  1995/10/06  23:05:13  kehr
%% Korrektur nach der Durchsicht von Karin.
%%
%% Revision 1.12  1995/09/22  01:12:02  kehr
%% Zweite �berarbeitung nch der inhaltlichen Korrektur. Au�erdem habe
%% ich das Logo zu MacIndex ver�ndert. Hat jetzt mehr pepp !
%%
%% Revision 1.11  1995/09/21  00:05:42  kehr
%% Erste Ver�nderungen nach der inhaltlichen Korrektur durch Joachim am
%% 20.Sep.95. Fast alle Dateien d'sind davon betroffen. Au�erdem sind noch zwei
%% neue Abbildungen hinzugekommen.
%%
%% Revision 1.10  1995/08/28  18:08:13  kehr
%% Neue Einspielung der xfig-Dateien
%%
%% Revision 1.9  1995/07/04  00:46:50  kehr
%% Bald ist's soweit ;-)
%% Ich habe heute die generelle Umstrukturierung vorgenommen und einige
%% Teile herausgeschmissen. Die Indexverarbeitung mu� noch �berarbeitet werden.
%%
%% Revision 1.8  1995/06/17  20:36:29  kehr
%% Habe die Lokationsreferenzverarbeitung umstrukturiert und besser
%% definiert. DIe Buchstabengruppen m�ssen noch beendet werden und der
%% Algorithmus zum Mischen und Sortieren der Lokationsreferenzen mu�
%% fertiggestellt werden.
%%
%% Revision 1.7  1995/06/09  20:59:49  kehr
%% Superviel gemacht heute ;-)
%%
%% Revision 1.6  1995/06/08  20:19:47  kehr
%% Bibliographie erweitert.
%%
%% Revision 1.5  1995/06/07  20:59:12  kehr
%% Und weiter am Modellentwurf. Spezifikation des Indexstyles vorerst
%% fertig. Es fehlt noch die Eingabegrammatik.
%%
%% Revision 1.4  1995/05/31  19:18:51  kehr
%% Fertigstellung des Analyse-Abschnitts (Hoffentlich ;-).
%%
%% Revision 1.3  1995/05/31  16:14:57  kehr
%% Dokumant- und Sortierungsalphabet entwickelt. Makefile�nderungen und
%% Stylever�nderungen.
%%
%% Revision 1.2  1995/05/29  00:22:43  kehr
%% Die Einleitung ist somweit fertig und die Analyse mu� jetzt noch
%% beendet werden. Mir fehlt da noch eine vern�nftige Idde f�r die
%% Alphabete und deren Definitionen.
%%
% Revision 1.1  1995/05/28  21:37:10  kehr
% Neue �berarbeitete Version.
% Inhaltliche �nderungen:
%   Glossar hinzugenommen. Einleitung mit Datenflu�graph. Kleinere
%   �nderungen an der Beschreibung des International Makeindex.
% System�nderungen:
%   Makefile-�nderungen, Stil�nderungen, Titelseite
%
%%

