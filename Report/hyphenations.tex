%%
%% $Id$
%%
%% Document: Praeambel mit Definitionen
%%

\hyphenation{%
  Lo-ka-tions-re-fe-renz
  Lo-ka-tions-re-fe-ren-zen
  In-de-xes In-de-xen
  Do-ku-ment-alpha-bet
  Do-ku-ment-alpha-bets
  Do-ku-ment-alpha-be-tes
  Sor-tier-alpha-bet
  Sor-tier-alpha-bets
  Sor-tier-alpha-bet-es
  Sor-tier-ungs-alpha-bet
  Sor-tier-ungs-alpha-bets
  Sor-tier-ungs-alpha-be-tes
  }

%% Local Variables:
%% mode: latex
%% TeX-master: "makeindex4.tex"
%% End:

%%
%% $Log$
%% Revision 1.1  1995/10/16 17:31:52  kehr
%% Initial checkin of Report and Presentation.
%%
%% Revision 1.21  1995/10/06  23:05:18  kehr
%% Korrektur nach der Durchsicht von Karin.
%%
%% Revision 1.20  1995/09/22  01:12:08  kehr
%% Zweite �berarbeitung nch der inhaltlichen Korrektur. Au�erdem habe
%% ich das Logo zu MacIndex ver�ndert. Hat jetzt mehr pepp !
%%
%% Revision 1.19  1995/09/21  00:05:48  kehr
%% Erste Ver�nderungen nach der inhaltlichen Korrektur durch Joachim am
%% 20.Sep.95. Fast alle Dateien d'sind davon betroffen. Au�erdem sind noch zwei
%% neue Abbildungen hinzugekommen.
%%
%% Revision 1.18  1995/09/06  18:52:52  kehr
%% Made final changes before giving for correction.
%%
%% Revision 1.17  1995/08/28  18:08:19  kehr
%% Neue Einspielung der xfig-Dateien
%%
%% Revision 1.16  1995/07/04  09:46:31  kehr
%% Weitere �nderungen. Bin aber fast fertig.
%%
%% Revision 1.15  1995/07/04  00:46:54  kehr
%% Bald ist's soweit ;-)
%% Ich habe heute die generelle Umstrukturierung vorgenommen und einige
%% Teile herausgeschmissen. Die Indexverarbeitung mu� noch �berarbeitet werden.
%%
%% Revision 1.14  1995/06/18  23:32:25  kehr
%% Schlu� f�r heute. Genug geschafft.
%%
%% Revision 1.13  1995/06/17  20:36:31  kehr
%% Habe die Lokationsreferenzverarbeitung umstrukturiert und besser
%% definiert. DIe Buchstabengruppen m�ssen noch beendet werden und der
%% Algorithmus zum Mischen und Sortieren der Lokationsreferenzen mu�
%% fertiggestellt werden.
%%
%% Revision 1.12  1995/06/13  21:55:17  kehr
%% Habe heute die Formulierung des Algorithmus controlled-jojo-traverse
%% fertiggestellt. Desweiteren Fehler in der Anwendung der \lindent-Umgebung
%% gefunden. Ich mu� noch die Matrix f�r die Definition der Ausgabekommandos
%% und der Angabe im Indexstyle entwickeln.
%%
%% Revision 1.11  1995/06/09  20:59:52  kehr
%% Superviel gemacht heute ;-)
%%
%% Revision 1.10  1995/06/08  20:19:48  kehr
%% Bibliographie erweitert.
%%
%% Revision 1.9  1995/06/06  23:50:15  kehr
%% Modellentwurf weitergebracht.
%%
%% Revision 1.8  1995/06/06  11:50:36  kehr
%% Weitere Bearbeitung des Modellentwurfs.
%%
%% Revision 1.7  1995/05/31  16:14:59  kehr
%% Dokumant- und Sortierungsalphabet entwickelt. Makefile�nderungen und
%% Stylever�nderungen.
%%
%% Revision 1.6  1995/05/28  21:37:12  kehr
%% Neue �berarbeitete Version.
%% Inhaltliche �nderungen:
%%   Glossar hinzugenommen. Einleitung mit Datenflu�graph. Kleinere
%%   �nderungen an der Beschreibung des International Makeindex.
%% System�nderungen:
%%   Makefile-�nderungen, Stil�nderungen, Titelseite
%%
%% Revision 1.5  1995/05/05  23:07:15  kehr
%% Ge�nderte Datenstrukturen mit enumerate und neuen labels f�r enumerate
%%
%% Revision 1.4  1995/05/05  22:25:06  kehr
%% Ge�nderte Struktur mit einleitung.tex
%% Zwischenspeicherung vor der Umstellung der Definitionen
%%
%% Revision 1.3  1995/04/30  16:14:11  kehr
%% Trennung in Einleitung, Einf�hrung und Analyse. Evtl. sollten die Filenamen
%% entsprechend anepa�t werden. Dar�berhinaus Analyse weitergef�hrt.
%%
%% Revision 1.2  1995/04/25  01:09:40  kehr
%% Analyse und Modellentwurf weitergebracht.
%%
%% Revision 1.1  1995/04/22  21:05:41  kehr
%% Erstes Setup der Studienarbeit des Makeindex4-Projektes
%%

%%% Local Variables:
%%% mode: plain-tex
%%% TeX-master: t
%%% End:
