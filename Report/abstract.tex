
\begin{abstract}

  Dieser Bericht beschreibt die Ergebnisse einer Studienarbeit am
  Institut f"ur Theoretische Informatik an der Technischen Hochschule
  Darmstadt in 1995. Ziel des Projektes war eine theoretische Analyse
  der Anforderungen an Indexierungssysteme und eine prototypische
  Implementierung.

  Indexierungssysteme verarbeiten die von einem Textsatzsystem
  erzeugten Indexierungsinformationen, um daraus einen sortierten und
  mit Ausgabeinformationen versehenen Index zu generieren. Dieser wird
  i.d.R.\ dann wieder dem Textsatzsystem zugef"uhrt. Gegenstand der
  Studienarbeit ist eine Analyse bestehender Systeme auf ihre
  M"oglichkeiten und eine darauf aufbauende Entwicklung eines
  Gesamtmodells der Indexverarbeitung. Besonderen Wert wurde dabei auf
  die Verarbeitung von Lokationsreferenzen und die Ausgabeformatierung
  gelegt. Hohe Benutzerkonfigurierbarkeit und Flexibilit"at waren
  weitere Ziele bei der Entwicklung. Eine Teilimplementierung der
  Kernaspekte des Modells wurde vorgenommen, um die wesentlichen
  Aspekte zu "uberpr"ufen.


\end{abstract}

