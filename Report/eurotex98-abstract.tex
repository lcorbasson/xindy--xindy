% $Id$
%----------------------------------------------------------------------
%
% xindy -- A Flexible Indexing System (Abstarct EuroTeX'98)
% Institut f�r Theoretische Informatik, TU Darmstadt
%
% [LaTeX2e]
% (history at end)

%\documentclass[titlepage,11pt]{article}
\documentclass[titlepage,12pt,draft]{article}

\hyphenation{
  strength-en
  Make-In-dex
  }

\usepackage{a4-9}
\usepackage{xspace}
\usepackage{environments}
\usepackage{ssqquote}
\usepackage{ifthen}
\usepackage{theorem}
\usepackage{bslash}
\usepackage{alltt}
\usepackage{enumerate}
\usepackage{verbatim}
\usepackage{afterpage}
\usepackage{rotating}
\usepackage{epsfig}
\usepackage[hang,sf]{caption}
%\usepackage[textures,bottomafter]{draftcopy}
%\usepackage{version}

\def\TechReport{1}
\def\ReportMode{0}
\usepackage{stdwrk}

\usepackage{xindy}

%\makeatletter
%\renewcommand{\@oddfoot}{\textsf{DRAFT}\hfil\textsf{\thepage}\hfil\textsf{DRAFT}}
%\makeatother

\frenchspacing

%%\def\mkidx{{\texttt{makeindex}}\xspace}
\def\mkidx{\emph{MakeIndex}\xspace}
\def\Xindy{{\normalfont\textsf{xindy}}\xspace}
\def\imkidx{{\emph{International MakeIndex}}\xspace}

\newcommand{\xindy}{%%
   \mbox{\normalfont%%
     \textsf{x\kern-0.6pt%%
       \shortstack{{\scriptsize$\circ$}\\[-2pt]\i}%%
       \kern-1pt%%
       ndy}%%}%%
     }\xspace}

\newcommand{\XINDY}{%
   \mbox{\normalfont\huge%%
         \textsf{x\shortstack{{\large$\circ$}\\\i}ndy}}}

%\renewcommand{\thefootnote}{\fnsymbol{footnote}}

\newlength\vertspace

\newenvironment{mfigure}[2]{%
  \begin{figure}[htbp]%
    \renewcommand{\tfigurecaption}%
                 {\caption{#1}}        % save the caption
    \renewcommand{\tfigurelabel}%
                 {\label{#2}}          % save the label
    \Hrule                             % start with horizontal line
    \vspace*{3mm}                      % small vertical space
}
{\vspace*{3mm}                         % another skip at the end
  \Hrule                               % another line
  \normalfont                          %
  \vskip\vertspace                     %
  \tfigurecaption                      % insert caption
  \tfigurelabel                        % and label
\end{figure}%                          % ok, we're done
}

\newenvironment{mtable}[2]{%
  \begin{table}[htbp]%
    \renewcommand{\ttablecaption}%
                 {\caption{#1}}        % save the caption
    \renewcommand{\ttablelabel}%
                 {\label{#2}}          % save the label
    \Hrule                             % start with horizontal line
    \vspace*{3mm}                      % small vertical space
}
{\vspace*{3mm}                         % another skip at the end
  \Hrule                               % another line
  \normalfont                          %
  \vskip\vertspace                     %
  \ttablecaption                       % insert caption
  \ttablelabel                         % and label
\end{table}%                           % ok, we're done
}

\begin{document}

\date{October 1997}
\title{\XINDY{}\\[1ex]%%
  A Flexible Indexing System}

\author{\textsc{Roger Kehr}\\[1ex]%%
  Darmstadt University of Technology\\[1ex]%%
  {\normalfont\normalsize\texttt{kehr@iti.informatik.tu-darmstadt.de}}}

\maketitle

\section*{Abstract}

In this article we present a new index processor named \xindy that has
been designed completely from scratch based on experiences with the
\mkidx~\cite{Chen:UCB-TR-87-347,Schrod:Makeindex30} and
\imkidx~\cite{Schrod:CG-10-81,Schrod:Makeindex30} systems.

\mkidx is probably the most used index processor in the \TeX{}
community. It covers a wide range of typical indexing problems and it
offers default settings that make it relatively easy to use. Despite
these advantages \mkidx has a lot of shortcomings. For example, it
lacks support for arbitrary alphabets, which is a problem for most of
the non-english speaking users. Additionally, its treatment of complex
location structures is implemented only on an ad-hoc basis without a
clean model.

The \imkidx has extended \mkidx with nomalization and sorting
mechanisms that support a general handling of the internationalization
problem. Experiences from this project have also shown that \mkidx is
hard to extend and further projects concerning indexing should start
from a well-defined model.

Based on these experiences we have developed a new indexing model that
gives a solid ground for further index processors.  This model has
contributed mostly to the concepts of \emph{locations} and
\emph{location references}. Locations are the entities the location
references of an index entry point to. Commonly used locations are
page numbers, section numbers, etc.\ New kinds of types---called
\emph{location classes}---can be composed of smaller entities like
arabic numbers, roman numerals, letters, etc.\ Examples of these
entities appear in books that have a page numbering scheme that starts
from 1 for each new chapter resulting in page numbers of the form
\emph{1-13, 2-15, 2-20,} etc. This type of locations can often be
found in computer manuals. A more complex structure is represented by
the locations \emph{Psalm~46, 1-8} and \emph{Genesis~1, 31}. The
structure of this kind of location references can be described in our
model and further processing such as sorting, merging and building of
ranges of location references is possible.

\xindy \cite{xindy,xindy:tr} is a full implementation of this model.
Though compatibility with \mkidx or \imkidx was not a design goal,
there exist tools and style files that make \xindy nearly behave as
the other processors. The internationalization part of \imkidx has
completely been incorporated into its implementation. Hence, it
subsumes all versions and features of \mkidx.

Another improvement is the \emph{markup backend}. Since an index
processor is only one component in a document preparation system, it
should fit smoothly into the rest of the environment. Our approach is
based on the concept of environments which has proven to be expressive
enough for most applications. The markup of an index can be defined in
a very comfortable, though extremely powerful way. \xindy comes with a
context-based markup strategy that uses a form of event dispatching
mechanism. This implements a very general scheme hat removes all the
existing limitations from the \mkidx systems and offers new dimensions
in the markup of indexes.


\section*{Presentation}

The presentation focuses less on the model, but more on real-life
examples that show how to use the basic concepts of \xindy. It
includes examples how to use the internationalization part, the
definition and processing of location classes, and a demonstration of
the markup backend. From a talk held at the DANTE\,'97 meeting in Munich
we think that 45 minutes are the minimum time to present \xindy. If
requested there is enough to tell for even an hour.


%%%%%%%%%%%%%%%%%%%%%%%%%%%%%%%%%%%%%%%%%%%%%%%%%%%%%%%%%%%%

\def\emdash{--}

\def\Lisp{{\normalfont\textsc{Lisp}}\xspace}
\def\STIL{{\normalfont\textsc{Stil}}\xspace}
\def\CL{{\normalfont\textsc{Common Lisp}}\xspace}
\def\CLOS{{\normalfont\textsc{Clos}}\xspace}
\def\dps{document preparation system\xspace}
\def\nroff{{\tt nroff}\xspace}
\def\term#1{\emph{#1}}
\def\pair#1{\mbox{$\langle$}#1\mbox{$\rangle$}}
\def\ts{\hspace*{0.9ex}}


%%%%%%%%%%%%%%%%%%%%%%%%%%%%%%%%%%%%%%%%%%%%%%%%%%%%%%%%%%%%

\bibliography{bibliographie}
\bibliographystyle{abbrv}

\end{document}

%%%%%%%%%%%%%%%%%%%%%%%%%%%%%%%%%%%%%%%%%%%%%%%%%%%%%%%%%%%%%%%%%%%%%%
%
% $Log$
% Revision 1.1  1997/11/25 19:06:07  kehr
% Initial checkin of both texts.
%
%
%
